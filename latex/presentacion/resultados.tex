\subsection{Caracterización de las palabras identificadas como contrastivas}

\begin{frame}[t]\frametitle{Palabras candidatas}

    \begin{columns}
        \begin{column}{.5\textwidth}
        Para buscar las palabras candidatas a tener contrastes significativos en cuanto a la cantidad de ocurrencias en distintas provincias, elegimos el conjunto de las primeras 
        cinco mil (5000) palabras con mayor valor de nuestra métrica.

        \end{column}

        \begin{column}{.45\textwidth}
            \begin{figure}
            \centering
            \includegraphics[width=0.95\linewidth]{../src/images2/valorInformacionCorte.pdf}
            \label{fig:ivalue}
            \end{figure}
        \end{column}
    \end{columns}

\end{frame}

\begin{frame}[t]\frametitle{Validación lingüística}

   \begin{itemize}

      \label{it:caracterizacionLinguistica}

      \item \textbf{Coloquialismos o vulgarismos}

      \blockquote[Córdoba]{Perdon pero tenes que ser muy \textbf{culiado/a} para ir a mc y pedirte una ensalada}


      \blockquote[Mendoza]{Q \textbf{chombi} hacer un chiste y q la otra persona no se ría o no lo entienda}

      %   \blockquote[Neuquén]{Que \textbf{carnasas} poniendole rosas rojas a toda la ropa, para mi queda horrible sorry}

      %   \blockquote[Chaco]{Teres, \textbf{pororós} y pelis con Carlita y Flor}

      % \blockquote[San Juan]{Ver un negro \textbf{chuño} con musculosa y gorro.. se ve que el tipo no quería pasar ni frío ni calor.}

      % \blockquote[Formosa]{Tenía la re expectativa para este sábado y al final \textbf{trancó} todo }

      \item \textbf{Indigenismos}

      \blockquote[Formosa]{Te regalo ser \textbf{mitaí} y ir a jurar la bandera con el guardapolvo caliente ese y la corbata que te ahorca todo (Del guaraní mitaí “pequeño”)}

      \blockquote[Corrientes]{\textbf{Angá} mi negrito, esta triste (Del guaraní angá aprox. “pobre”)}

      \item \textbf{Gentilicios}

      \textbf{Casildense} (de Casilda), \textbf{concordiense} (de Concordia) y \textbf{obereño} (de Oberá).

\end{itemize}

% \item \textbf{Voces no marcadas en registro, que aluden a una realidad local}

%   \blockquote[San Juan]{Quiero a alguien que me diga vamos a comer \textbf{piadinas}, un pancho, un chori, una hamburguesa lo que sea y soy feliz}

%   \blockquote[Misiones]{\textbf{Tareferos} que reclamaban asistencia interzafra en Posadas estarían preparando una protesta para hoy en la Fiesta del Inmigrante en Oberá.}

%   \blockquote[Jujuy]{Me encantan los bohemios anti sistema que usan vans. Es como que seas ecologista y uses un cuaderno hecho con media \textbf{yunga}.}

\end{frame}

\begin{frame}[t]\frametitle{Más resultados}
    

\begin{itemize}
  
\item \textbf{Leísmo}

  \blockquote[Misiones]{No te olvides de \textbf{saludarle} a tu suegro hoy}

  \blockquote[Misiones]{Vine a \textbf{visitarle} a mis primas y estan re colgadas, para eso me quedaba en mi casa no maaa }

  \blockquote[Formosa]{A \textbf{esperarle} a nahuel, que traiga los teresss }

% \item \textbf{Fusiones y acrónimos que pueden señalar pronunciación o alta frecuencia de uso}

%   \blockquote[Buenos Aires]{Los sueños de la siesta me dejan \textbf{patra} }

%   \blockquote[Córdoba]{Si mañana me dice q no, voy sola, necesito ver esa pelicula en el cine siosi}

% % \item \textbf{Voces consideradas generales pero que, al aparecer en la lista, permitieron verificar su contrastividad en frecuencia de uso al menos con respecto a España}
%   % Ejemplos: \textbf{pavada}, \textbf{distrital} y \textbf{cariño}.

\item \textbf{Voces sospechadas generales pero con acepción local diferente}

  \blockquote[Mendoza]{Mañana que alguien \textbf{atine} con parque y porrones}

  \blockquote[San Juan]{\textbf{Mansas} ganas de sentarme a tomar un te con semitas}

  \blockquote[Tierra del Fuego]{\textbf{Habilítenme} una nueva espaldaa}

  \blockquote[San Juan]{sigo \textbf{asada} por cosas que han pasado hace como dos dias, que falla (Mendoza) / Que \textbf{asada} estoy, tengo la cabeza echa un lío}


\item \textbf{Voces con una morfología propia de una región}

Ejemplo: terminación azo/aza con base adjetiva.

  \blockquote[San Juan]{Creo que va a estar \textbf{malazo} lo de esta noche } 

  \blockquote[Córdoba]{Esta \textbf{locaza} esa mina para hacer eso}

% \item \textbf{Variantes ortográficas}

%   Ejemplos: culiado (adj. despect. o fórmula de tratamiento de confianza) y tereré.

%   \blockquote[Tucumán]{Menos mal que soy de los chetos de la carne y mañana tengo \textbf{asao} todo el dia jajajajaj}

%   \blockquote[Catamarca]{Un lunes con buen humor ta \textbf{pasao} }

%   \blockquote[Corrientes]{Ahora a la mañana tengo q ir hacerme la tarjebus jajajajj \textbf{mavale} q me estoy por levantarrr jajajaj}  

%   \blockquote[Córdoba]{Q paja volver al colegio \textbf{culiaa}}

%   \blockquote[Córdoba]{Que pajero el \textbf{qliao} este.}

%   \blockquote[Córdoba]{Quiero recitaaal \textbf{qliaaaa}}

%   \blockquote[Entre Ríos]{\textbf{Tereresss} y pile con todos mis primisss}

%   \blockquote[Corrientes]{No se si hacerme un \textbf{tere} o un mate para pasar la siesta}

%   \blockquote[Chaco]{Es lo mas lindo no ir al colegio y quedarme a tomar \textbf{teresss}}


% \item \textbf{Formas verbales coloquiales con sustantivos o adjetivos como base}

%   \blockquote[Neuquén]{Me calma mucho \textbf{mimosear} a mi perro }

%   \blockquote[Buenos Aires]{Me vine a acostar y ya me dicen que parezco de 80 años ME CHUPA UN HUEVO LO QUE PIENSEN, DEJENME \textbf{ABUELEAR} }

%   \blockquote[Tierra del Fuego]{Estaría bueno que ari venga aunque sea a saludarme y que no se quede todo el tiempo \textbf{pollereando}.}


% \item \textbf{Vesres}: Creación de palabras por inversión de sílabas que se usa jergalmente o con fines humorísticos.

%   \blockquote[Corrientes]{Estoy en lo de villa mateando con él y jimmy. Pinta \textbf{sogui} abundante más tarde dijeron }

%   \blockquote[Chaco]{Uhhh me acuerdo si no habré saltado el muro del aguapey par colarme a los \textbf{cequin}. (cequín “fiesta de quince”)}

% \item \textbf{Intejercciones}

%   \blockquote[Formosa]{\textbf{Aijué}, encima me decís vieja, re que no pinta esto facundo jaja ya te dije como es la onda, fin }

%   \blockquote[Formosa]{\textbf{Ains}, una mujer hablando de fútbol.}

%   \blockquote[Corrientes]{Al fin una buena: hora libreeee! \textbf{Yirr} }
% \end{itemize}

\end{itemize}
\end{frame}


\subsection{Validación estadística}

\begin{frame}[t]\frametitle{Problema desde el punto de vista estadístico}


    


\end{frame}

\begin{frame}[t]\frametitle{Test hipergeométrico}

    Para aplicar el test hipergeométrico representamos los datos sobre la palabra en una tabla de 2x2 como la de la siguiente Tabla.

    \begin{table}[ht]
    \centering
    \begin{tabular}{l|ccc}
    \hline
    &  \begin{tabular}{@{}c@{}}\#Palabras \\sobre región\end{tabular} &  \begin{tabular}{@{}c@{}}\#Palabras en el \\ resto de Argentina\end{tabular}  &Total \\ \hline
    \# Palabras w &   $k$ & $K-k$ & $K$ \\ 
    \# Palabras $\neq$ w & $n-k$ & $N + k - n - K$  & $N - K$ \\ \hline
    Total & $n$ & $N -n$ & $N$ \\ 
    \end{tabular}
    \label{tab:contingencia}
    \end{table}

\end{frame}

\begin{frame}[t]\frametitle{Test t de Welch}

    El test de Welch nos provee un valor de probabilidad para rechazar la hipótesis nula que afirma que las medias de las dos distribuciones son iguales.

    \only<2>{
       Las suposiciones del test 
        \begin{enumerate}
            \item Todos los textos son estadísticamente independientes 
            \item La media de las frecuencias proviene de una distribución normal
        \end{enumerate}
       } 
    
    \only<3>{
       \begin{block}{Metodología}
              Agrupamos todos los tuits de cada usuario representando un texto.\footnote{Notar que la suposición de independencia es más débil.}
        \begin{description}
            \item[Corpus S] Todos los textos de los usuarios que provienen de las provincias en donde se cubre el 80\% de las ocurrencias
            \item[Corpus T] Los textos creados por usuarios del resto de las provincias
        \end{description}

        \end{block}
       }
\end{frame}


\begin{frame}[t]\frametitle{Resultados test t de Welch}

  \begin{enumerate}
    \alert<1>
    {
    \item Para cada métrica $I,I_W,I_P$  variamos los subconjuntos de palabras de acuerdo al listado ordenado según estas.
    }
    
    \alert<2>
    {
    \item Calculamos la tasa de rechazo de la hipótesis nula, definida por:
    \begin{equation}
      \text{Tasa de rechazo(tests)} = \frac{\# \{t:tests \mid p-valor(t) < 0.05\}}{\#tests} 
    \end{equation}
    } 
  \end{enumerate}

  \begin{figure}
  \includegraphics[width=0.50\linewidth]{../src/images/rechazo_metricas.pdf}
            % \label{fig:rechazo_metricas} 
  
  \end{figure}
            
\end{frame}