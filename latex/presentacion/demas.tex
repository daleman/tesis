\begin{frame}[t]\frametitle{Test hipergeométrico}

    Para aplicar el test hipergeométrico representamos los datos sobre la palabra en una tabla de 2x2 como la de la siguiente Tabla.

    \begin{table}[ht]
    \centering
    \begin{tabular}{l|ccc}
    \hline
    &  \begin{tabular}{@{}c@{}}\#Palabras \\sobre región\end{tabular} &  \begin{tabular}{@{}c@{}}\#Palabras en el \\ resto de Argentina\end{tabular}  &Total \\ \hline
    \# Palabras w &   $k$ & $K-k$ & $K$ \\ 
    \# Palabras $\neq$ w & $n-k$ & $N + k - n - K$  & $N - K$ \\ \hline
    Total & $n$ & $N -n$ & $N$ \\ 
    \end{tabular}
    \label{tab:contingencia}
    \end{table}

\end{frame}



\begin{frame}[t]\frametitle{Metrica}
\label{fr:metrica}
% Despues definis el n\'umero minimo y m\'aximo  en el que ocurre cada palabra a lo largo de todo el pa\'is, en escala logartimica. Para ello, definimos 
\begin{equation}
\label{cant_W_def}
cant_W(w)=\log(\mathbf N(w)),
\end{equation} 
y 

\begin{equation}
\label{max_min_pais}
MIN_W=\min_{w\in \text{Palabras}} cant_W(w)
\;,\quad MAX_W=\max_{w\in \text{Palabras}} cant_W(w)
\end{equation}
\begin{block}{Índice normalizado}
\begin{equation}
\label{norm_W_def}
norm_W(w)=\frac{cant_W(w)-MIN_W}{MAX_W-MIN_W}
\end{equation}
Esto se define para palabras $w$ con $\mathbf{w}\geq 40$. 
\end{block}

\hyperlink{valoresContrastivos}{\beamerbutton{Volver}}

\end{frame}