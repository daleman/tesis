
\begin{frame}[t]\frametitle{Qué es una palabra contrastiva}
    
    Se dice que una palabra es \textit{contrastiva} cuando la frecuencia de uso en distintas regiones es muy diferente. 

    \begin{block}{Ejemplos palabras contrastivas Argentina - España}
    \begin{itemize}
        \item ``che''
        \item ``metegol''
    \end{itemize}
    \end{block}

    \begin{block}{Ejemplos palabras contrastivas dentro de Argentina}
    \begin{itemize}
        \item ``gurisada''
    \end{itemize}
    \end{block}

¿Para qué sirve conocer las palabras contrastivas?

\end{frame}

\begin{frame}[t]\frametitle{Motivación de un método computacional para detectar léxico contrastivo}
    
¿Cómo se conocían las palabras contrastivas?
Mediante encuestas.
\begin{itemize}
    \item Es costoso de realizar
    \item Muy difícil de hacer de forma balanceada en distintas regiones de un país o de un continente.
    \item Más difícil es encuestar a una gran cantidad de personas.
    \item \alert{Se basan en el conocimiento \textit{a priori}}
\end{itemize}
\end{frame}