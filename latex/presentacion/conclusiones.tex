\begin{frame}[t]\frametitle{Conclusiones}
    
    \begin{itemize}
        \item Desarrollamos una métrica de la contrastividad de uso de una palabra en distintas regiones.
        \item Para probar esta métrica recolectamos un conjunto de datos de textos de la Argentina a través de la API de Twitter.
        \item Obtuvimos aproximadamente 1 palabra contrastiva relevante lingüísticamente cada 17 palabras.
        \item Varias de las palabras detectadas a partir de la métrica desarrollada serán agregadas al Diccionario del habla de los argentinos.
    \end{itemize}

\end{frame}

\begin{frame}[t]\frametitle{Trabajo a futuro}
    
    \begin{itemize}
        \item Reproducir el trabajo para todos los países hispanoparlantes.
        \item Obtener regiones dialectales a partir de métodos de clustering, lo cual permitiría validar la vigencia de las regiones propuestas por Vidal de Battini en 1964 \cite{vidal1964espanol}.
        \item Analizar la contrastividad léxica comparando la distribución de n-gramas.
    \end{itemize}

\end{frame}

\begin{frame}[t]\frametitle{¿Preguntas?}
    
\end{frame}