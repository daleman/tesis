
\subsection{Primeras métricas: MaxDif y  \texorpdfstring{$MaxDif_g$}{MaxDifg}} 



% \begin{center}
%    \begin{tabular}{ l | c || r | }
%      \hline
%      1 & 2 & 3 \\ \hline
%      4 & 5 & 6 \\ \hline
%      7 & 8 & 9 \\
%      \hline
%    \end{tabular}
%  \end{center}
    \begin{frame}[t]\frametitle{Primeras métricas: MaxDif}
Para cada palabra $\omega$ y cada par de provincias $p_1$ y $p_2$:
% , podemos calcular el cociente entre la frecuencia máxima de $\omega$ en ambas provincias y la frecuencia mínima:

\begin{equation}
  \label{eq:maxDif} 
  maxDif(\omega,p_1,p_2) = \frac{\text{frec.\ m\'axima de $\omega$ en ambas provincias}}{\text{frec.\ m\'inima de $\omega$ en ambas provincias}}
\end{equation}

\only<1>{
Desventajas:  
\begin{enumerate}
    \item \label{1} Un valor para cada par de provincias.
    \item \label{2} No se considera la dispersión de los valores en todas las provincias.
\end{enumerate}
}
    
% hago un vector de 23 frecuencias, coloreo la más baja y la más alta y las divido

\only<2>{
\begin{block}{$MaxDif_g$}
Considerando las frecuencias de una palabra $\omega$ sobre todas las provincias:

\begin{equation}
 maxDif_g(\omega) = \frac{\text{frec.\ m\'axima de $\omega$ \textbf{todas las provincias}}}{\text{frec.\ m\'inima de $\omega$ sobre \textbf{todas las provincias}}}
 \label{eq:maxDifg}  
\end{equation} 
\end{block}  

\begin{enumerate}
    \item Se resume en un único valor la contrastividad de la palabra
    \item Sigue sin considerar la distribución de las frecuencias
\end{enumerate}

}

\end{frame}

% \begin{frame}[t]\frametitle{Primeras métricas: }

% Considerando las frecuencias de una palabra $\omega$ sobre todas las provincias, definimos:

% \begin{equation}
%  maxDif_g(\omega) = \frac{f_{max}^\prime(\omega)}{f_{min}^\prime(\omega)}
%  \label{eq:maxDifg}  
% \end{equation} 
% % donde $f_{max}^\prime(\omega)$ es la frecuencia máxima de la palabra $\omega$ entre las frecuencias de todas las provincias y $f_{min}^\prime(\omega)$ es la frecuencia mínima distinta de $0$.

% \begin{itemize}
%     \item Se resume en un único valor la contrastividad de la palabra
%     \item Sigue sin considerar la distribución de las frecuencias
% \end{itemize}

% \end{frame}

\subsection{Entropía y valor de la información}

\begin{frame}[t]\frametitle{La entropía de la información}
    
% La entropía nos brinda un valor que indica qué tan uniforme es la distribución de las frecuencias de cada palabra. 
\only<1->{
    Sea $\mathbf{p}=(p_1,\ldots, p_n)$ un vector de probabilidad puntual: $p_i\geq 0$ y $\sum_{i=1}^np_i=1$. 

            Definimos la entropía de $\mathbf{p}$ siendo
            \begin{equation}
                H(\mathbf{p})=-\sum_{i=1}^n \log(p_i) p_i
            \end{equation}    
}

\only<2>{
    \begin{columns}
    \begin{column}{.30\textwidth}
        \begin{figure}
        \includegraphics[width=0.9\textwidth]{../src/images/presentacion/trump.jpg}
        \caption{Poco predecible, alta entropía} 
        \label{fig:trump}
        \end{figure}
    \end{column}
    \begin{column}{.30\textwidth}   
        \begin{figure}
        \includegraphics[width=0.9\textwidth]{../src/images/presentacion/coyote.png}
        \caption{Muy predecible, baja entropía} 
        \label{fig:coyote}
        \end{figure} 
    \end{column}
    
    \end{columns}

}

\only<3>
{
    \begin{exampleblock}{Dime que tan uniforme eres y te diré cuanta entropía tienes} Las palabras utilizadas más uniformemente en las distintas provincias como \textit{de} o \textit{que} aportan menos información que la palabra \textit{chomaso}. 
\end{exampleblock}
}

\only<4>
{
    \begin{block}{Observaciones}
        \begin{enumerate}
            \item Entropía mínima ($=0$)  si y solo si $\mathbf{p}$ esta concentrada en un unico punto: existe $i$ tal que $p_i=1$ y $p_j=0$, para todo $j\not=i$. 
            \item Entropía máxima con $\mathbf{p}$ equiprobable: $p_i=1/n$, para todo $i$. 

        \end{enumerate}

    \end{block}
}
\end{frame}

\begin{frame}[t]\frametitle{Valor de la información}

Zanette y Montemurro definieron al \textit{valor de la información de una palabra} como
 \begin{equation}
  \Delta I(\omega) = p(\omega) \,  (\widehat{H}(\omega) - H(\omega))  =  p(\omega) \, \Delta{H(\omega)}
 \end{equation}
siendo $p(\omega)$ la frecuencia total de la palabra en el texto, $p(\omega) = n/N$ 

N = \# palabras en texto,
n = \# ocurrencias de $\omega$

\begin{equation}
 n_1 \ldots n_p \rightarrow H(\omega)   
\end{equation}



\end{frame}

\subsection{Valores contrastivos}
\begin{frame}[t]\frametitle{Valor contrastivo sobre las palabras}
\label{valoresContrastivos}
    \begin{alertblock}{Valor contrastivo sobre las palabras}
        \begin{equation}
            I_w(\omega) = norm_{w}(\omega) \cdot (\widehat{H}_{w}(\omega) - H_{w}(\omega))
            \label{eq:iw}
        \end{equation}
        donde \hyperlink{fr:metrica}{$norm_w$} sirve para normalizar sobre la cantidad de ocurrencias de la palabra.
    \end{alertblock}

    \begin{block}{Observaciones}
        \begin{itemize}
            \item Si una palabra se dice muchas veces el valor de $I_w(\omega)$ es más alto.
            \item Si dos palabras se dicen la misma cantidad de veces, la palabra que tenga una dispersión más heterogenea será la de mayor valor contrastivo sobre las palabras.
        \end{itemize}
    \end{block}

\end{frame}

\begin{frame}[t]\frametitle{Robusteciendo métrica: valor contrastivo sobre las personas}
    \only<1>{
    ¿Y si algunas palabras tienen un $I_w$ alto debido a pocas personas que las mencionan constantemente?
    \medskip
    
    \alert{Hay que tener en cuenta a la distribución de la cantidad de personas que mencionan cada palabra en las provincias.}
    }

    \only<1-2>{
        \begin{alertblock}{Valor contrastivo sobre las personas}
        \begin{equation}
            I_p(\omega) = norm_p(\omega) \cdot (\widehat{H}_p(\omega) - H_p(\omega))
            \label{eq:iu}
        \end{equation}
        donde $norm_p$ sirve para normalizar sobre la cantidad de personas que mencionan la palabra $\omega$.
    \end{alertblock}
    }

    \only<2>{
        \centering
        \includegraphics[width=.55\textwidth]{../src/images2/entropiaPersonasxNormCantPersonasColor_sinFiltro.pdf}
    }

    
\end{frame}

\begin{frame}[t]\frametitle{Valor de contrastividad}
    \begin{block}{Valor de contrastividad}
    Como nos interesa tanto la distribución de la cantidad de ocurrencias de cada palabra, como la de la cantidad de usuarios que la menciona, definimos   
    \begin{equation}
    I(\omega) =  I_w (\omega) \cdot I_p(\omega)
    \label{eq:ivalor}
    \end{equation}
    \end{block}    

    % \begin{block}{Palabras candidatas}

    % \begin{columns}
    %     \begin{column}{.5\textwidth}
    %     Para buscar las palabras candidatas a tener contrastes significativos en cuanto a la cantidad de ocurrencias en distintas provincias, elegimos el conjunto de las primeras 
    %     cinco mil (5000) palabras con mayor valor de nuestra métrica.

    %     \end{column}

    %     \begin{column}{.45\textwidth}
            \begin{figure}
            \centering
            \includegraphics[width=0.55\linewidth]{../src/images2/valorInformacionCorte_sinFiltro.pdf}
            % \caption{Palabras Candidatas}
            \label{fig:ivalue}
            \end{figure}
    %     \end{column}
    % \end{columns}

% \end{block}
    

\end{frame}

\begin{frame}[fragile]\frametitle{Proporción de ocurrencias}
    % Detectadas las palabras que se utilizan en forma diferenciada, interesa detectar donde se utilizan mas.
    ¿Dónde se utiliza más?
\begin{verbatim}
por cada palabra w:
  por cada k = 1..23:
    calculamos la proporción acumulada de ocurrencias 
    por las k provincias que más mencionan a w
\end{verbatim}
\begin{columns}
\begin{column}{.5\linewidth}
\centering
\includegraphics[width=1.1\linewidth]{../src/images/PropAcum5000SinCandidatas2.pdf}
\end{column}
\begin{column}{.5\linewidth}
\centering
\includegraphics[width=1.1\linewidth]{../src/images/PropAcumSinCandidatas2.pdf}
\end{column}
\end{columns}
% \centering 
% \begin{columns}
%         \begin{column}{.7\linewidth}
%         \end{column}
%         \begin{column}{.3\linewidth}
%         Proporción de ocurrencias acumulada según la muestra de palabras. % El número de la leyenda indica la cantidad de palabras contrastivas elegidas para la muestra respectiva, siempre seleccionando las más contrastivas según la métrica.
%         \label{fig:propAcum}
%         \end{column}
%       \end{columns}
           

\end{frame}