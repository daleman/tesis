
\begin{frame}[t]\frametitle{Web como un corpus}
    % Kilgariff comentaba la riqueza de la Web para acceder a un volúmen de información, antes impensado.
    Kilgariff: riqueza de la Web, muchos datos.
    %Principalmente se hacían estudios estimando la frecuencia de una palabra viendo la cantidad de resultados de las consultas en un motor de búsqueda.

\bigskip
\begin{columns}
    \begin{column}{0.5\textwidth}
        \centering{\textbf{Ventajas}}
        \begin{itemize}
            \item Gratuito
            \item Gran cantidad de datos
            \item Disponible e inmediato
        \end{itemize}
    \end{column}
    \begin{column}{0.5\textwidth}
        \centering{\textbf{Utilidades}}
        \begin{itemize}
            \item Corrección de ortografía
            \item Traducción de frases
            \item Estimar el tamaño de la Web
        \end{itemize}
    \end{column}
\end{columns}

\bigskip
\begin{block}{Cita}
\textcquote[p. 342]{kilgarriff2003introduction}{La web es un corpus sucio, pero el uso esperado es mucho más frecuente que lo que puede considerarse como ruido.} 
\end{block}

\end{frame}