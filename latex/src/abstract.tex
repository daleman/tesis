\begin{abstract}
El crecimiento de la cantidad de datos en la web y los recursos computacionales de las últimas décadas da la posibilidad de investigar fenómenos lingüísticos a gran escala, 
tarea casi imposible de realizar manualmente. En el área de la lexicografía se conoce como palabras contrastivas, a aquellas que poseen una frecuencia de uso significativamente distinta en dos regiones. 
En el presente trabajo desarrollamos una métrica para detectar estas palabras. Esta hace uso de la entropía de la información sobre la cantidad de ocurrencias de cada palabra y la cantidad de usuarios que la utilizan para dar un indicio de la contrastividad. Para evaluarla se recolectó a partir de Twitter un conjunto de datos con más de 190 millones de palabras escritas y más de 20 mil usuarios de las 23 provincias argentinas. 
Este trabajo multidisciplinario se hizo en colaboración de la Academia Argentina de Letras, la cual realizó la validación y relevó una tasa de 300 palabras contrastivas con relevancia lingüística. Varias de ellas formarán parte de las palabras del Diccionario del Habla de los Argentinos del año próximo. 
\end{abstract}
