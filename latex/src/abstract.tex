\begin{abstract}
El crecimiento de la cantidad de datos en la web y los recursos computacionales de las últimas décadas dan la posibilidad de investigar fenómenos lingüísticos a gran escala, 
tarea casi imposible de realizar manualmente. En el área de la lexicografía se conoce como palabras contrastivas, a aquellas de una misma lengua que tienen una frecuencia de uso significativamente distinta en dos o más regiones. 
En el presente trabajo desarrollamos una métrica para detectar estas palabras. Esta hace uso de la entropía de la información sobre la cantidad de ocurrencias de cada palabra y la cantidad de usuarios que la usan para dar un indicio de la contrastividad. Para evaluarla se recolectó a partir de Twitter un conjunto de datos con más de 190 millones de palabras escritas y más de 20 mil usuarios de las 23 provincias argentinas. 
Este trabajo multidisciplinario se hizo en colaboración de la Academia Argentina de Letras, la cual realizó la validación y comprobó diferencias relevantes en la extensión de uso de unas 300 palabras. Varias de ellas formarán parte de las palabras del Diccionario del Habla de los Argentinos del año próximo. 

% \textbf{Palabras claves:} Procesamiento del lenguaje natural, Lingüística computacional, Lexicografía contrastiva

\end{abstract}
