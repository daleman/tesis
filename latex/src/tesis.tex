\documentclass[11pt,a4paper]{tesis}
% SI NO PENSAS IMPRIMIRLO EN FORMATO LIBRO PODES USAR
%\documentclass[11pt,a4paper]{tesis}

\usepackage{graphicx}
\usepackage{adjustbox}
\usepackage{amssymb}
\usepackage{amsmath}
\usepackage{amsthm}
\usepackage{booktabs}
\usepackage[tables]{xcolor}
\usepackage[obeyFinal]{easy-todo}
\usepackage{colortbl}
\usepackage[utf8]{inputenc}
\usepackage[spanish]{babel}
\usepackage[left=3cm,right=3cm,bottom=3.5cm,top=3.5cm]{geometry}
\usepackage{xspace}
\usepackage{caption}
\usepackage[labelformat=simple]{subcaption}
\renewcommand\thesubfigure{(\alph{subfigure})}
\usepackage{url}
\usepackage[bookmarks=true]{hyperref}
\hypersetup{
             % show bookmarks bar?
    unicode=true,          % non-Latin characters in Acrobat’s bookmarks
    pdftoolbar=true,        % show Acrobat’s toolbar?
    pdfmenubar=true,        % show Acrobat’s menu?
    pdffitwindow=false,     % window fit to page when opened
    pdfstartview={FitH},    % fits the width of the page to the window
    pdftitle={Tesis Damián Aleman},    % title
    pdfauthor={Damián Eliel Aleman},     % author
    pdfsubject={Subject},   % subject of the document
    pdfcreator={Creator},   % creator of the document
    pdfproducer={Producer}, % producer of the document
    pdfkeywords={Tesis, Linguistica computacional, Procesamiento del lenguaje natural, Lexicografía contrastiva}, % list of keywords
    pdfnewwindow=true,      % links in new PDF window
    colorlinks=true,       % false: boxed links; true: colored links
    linkcolor=black,          % color of internal links (change box color with linkbordercolor)
    citecolor=blue,        % color of links to bibliography
    filecolor=magenta,      % color of file links
    urlcolor=cyan,           % color of external links
    pageanchor=true
}
\usepackage[style=american]{csquotes}
\pdfsuppresswarningpagegroup=1

\begin{document}

%%%% CARATULA
% Comentar y descomentar según corresponda
\def\titulo{Licenciatura\xspace}

\def\autor{Damián Eliel Aleman}
\def\tituloTesis{Hacia un método computacional para detectar léxico contrastivo}
\def\runtitulo{\tituloTesis}
\def\director{Juan Manuel Pérez y Santiago Kalinowski}
\def\codirector{Agustín Gravano}
\def\lugar{Buenos Aires, 2017}
%\input{my_macros}

\input{caratula}

%%%% ABSTRACTS, AGRADECIMIENTOS Y DEDICATORIA
\frontmatter
\pagestyle{empty}

\noindent A Edgar Altszyler por su gran ayuda para el desarrollo de la métrica para detectar el léxico contrastivo.\\

\noindent Al equipo de lexicógrafos de la Academia Argentina de Letras que analizaron de forma pormenorizada las palabras candidatas a ser contrastivas, entre ellos Pedro Rodríguez Pagani, Gabriela Pauer, María Sol Portaluppi y Josefina Raffo.\\

\noindent A Federico Plager y a Mariela Sued, por tomarse el tiempo y la dedicación de leer esta tesis y por sus devoluciones.\\

\noindent A mis directores, a Juan Manuel Pérez por todas las reuniones y el acompañamiento constante, a Santiago Kalinowski por acercarme a la lingüística y a Agustín Gravano por sus aportes para entender un poco más el proceso de investigación.\\

\noindent A mi mamá, mi papá y mis hermanos por estar siempre.\\

\noindent A Juli, a Edu y al abuelo.\\

\noindent A los amigos de la facu: Martín, Javier, Luis, Guille, Fixman, Ralo, JP ,Diego y Guerson.\\

\noindent A Martín, por ser compañero de viajes, de antidomingos y de toda la carrera.\\

\noindent A mis compañeros de Hacoaj, especialmente a Gabi y Eio.\\

\noindent A mis compañeros del Dari.\\
\begin{abstract}
El crecimiento de la cantidad de datos en la web y los recursos computacionales de las últimas décadas dan la posibilidad de investigar fenómenos lingüísticos a gran escala, 
tarea casi imposible de realizar manualmente. En el área de la lexicografía se conoce como palabras contrastivas, a aquellas de una misma lengua que tienen una frecuencia de uso significativamente distinta en dos o más regiones. 
En el presente trabajo desarrollamos una métrica para detectar estas palabras. Esta hace uso de la entropía de la información sobre la cantidad de ocurrencias de cada palabra y la cantidad de usuarios que la usan para dar un indicio de la contrastividad. Para evaluarla se recolectó a partir de Twitter un conjunto de datos con más de 190 millones de palabras escritas y más de 20 mil usuarios de las 23 provincias argentinas. 
Este trabajo multidisciplinario se hizo en colaboración de la Academia Argentina de Letras, la cual realizó la validación y comprobó diferencias relevantes en la extensión de uso de unas 300 palabras. Varias de ellas formarán parte de las palabras del Diccionario del Habla de los Argentinos del año próximo. 

% \textbf{Palabras claves:} Procesamiento del lenguaje natural, Lingüística computacional, Lexicografía contrastiva

\end{abstract}


\tableofcontents

\mainmatter
\pagestyle{headings}

%%% Macros

%%%% ACA VA EL CONTENIDO DE LA TESIS
\chapter{Introducción}
\label{ch:introduccion}

%Explicación del área
%el trabajo de vidal de batini y almeida
%Objetivo de la tesis

%Contrsuccion de lingüistica de corpus
%lingüistica computacional, su evolucion
%Todo se hace en ingles y hay pocos analisis en castellano.

%primer acercamiento de contrastes léxicos 


%Comentar el problema de la lingüística de Corpus, trabajo previo en el área, comentar brevemente qué queremos hacer. Acá va a ir mucho de lo que nos pasó Santiago como literatura, incluso mencionando (por ejemplo) el trabajo de Vidal de Battini
\subsection{Trabajo previo en el área}
\par Actualmente, las herramientas para la detección de palabras con contraste léxico en distintas regiones
consisten en cuestionarios como el de \emph{Almeida}\cite{almeida1995variacion}.  Estos cuestionarios están integrados por grupos temáticos centrales como la casa, la familia, la enseñanza, el cuerpo humano, etc. Sobre cada grupo temático se les indicaba a las personas entrevistadas un repertorio de palabras por cada noción, para ver si las conocían y con que frecuencia se las usaba. 
Con este trabajo planteamos cambiar el paradigma y detectar automáticamente las palabras usadas en distintas regiones y sus frecuencias.

Uno de los corpus lingüísticos del español más reconocidos es el \emph{corpes XXI}\cite{espanolabanco}, creado por la Real Academia Española con una distribución de 25 millones de formas por cada uno de los años comprendidos en el periodo 2001 
a 2012. Sin embargo, dicho corpus tiene dos desventajas importantes: por un
lado, la cantidad de palabras de América Latina están subrepresentados ya que el $65,70$\% de las palabras del dataset provienen de textos de España y $34,30$\% de los demás países hispanoparlantes. Por otro lado, uno no dispone de todo el dataset, sino que solamente se pueden hacer las consultas de su página web. Estas consultas están limitadas en cuanto a la cantidad de solicitudes y a las funcionalidades que estas proveen.

Una de las virtudes de hacer un corpus con un método de recolección de textos de forma automática se desprende de la cantidad superadora de longitud del corpus en comparación con métodos manuales como digitalización de textos.

A pesar de haber comenzado hace varias decadas la recolección de textos de la web para realizar corpus, no hay muchos en el idioma español.
Uno que se pudo encontrar es el de \textit{Mark Davies}, el cual se utilizó las páginas web para recolectar los textos, con dos billones de palabras en español y divide a las páginas en regiones a través de  

Las ventajas de Twitter son claras: da una interfaz pública para obtener tweets de cualquier persona. Además a diferencia de un portal de noticias donde los comentarios suelen estar relacionados con estas, en Twitter son más amplio los tópicos de los comentarios.
Por otro lado, Twitter es una tecnología que permite hacer escalabale el trabajo a diferentes paises ya que con una misma interfaz se pueden obtener los textos de cualquier región. En cambio, si se elige un portal de noticias para sacar comentarios de usuarios, deberíamos ver la estructura de cada página para obtener esos datos.
Otra ventaja de Twitter es que cada usuario está identificado y se podría llegar a inferir datos como género, edad y ubicación de cada uno.
Por último una ventaja sobre otro método de recolección, es que a través de Twitter se puede recolectar textos con una granularidad regional muy variable y obtener información de quienes lo escriben.
Existen otras redes sociales que se podrían obtener textos para analizar , pero tienen la desventaja de ser privadas como Facebook o ser acotadas en términos de los temas que se hablan en la misma, un caso de esto es Linkedin. 

Para dar una noción de la cantidad de usaurios en la Argentina:
En el 2016 había 11,8 millones de usuarios de Twitter en la Argentina, con 15 millones de personas con smartphones, lo cual el 70 de la gente con smartphones tenía Twitter.

\subsection{Lingüistica computacional} % (fold)
\label{sub:linguistica_computacional}



\chapter{Datos: Extracción y procesamiento}
\label{ch:datos}
%Acá explicamos cómo extrajimos los datos, qué características tiene la muestra (muchos de los análisis/gráficos ya los hicimos), cómo la separamos, y qué análisis estadísticos estamos realizando (es la parte que nos falta)


\section{Extracción de Datos}

Para la recolección de tuits, primero se extrajo una cantidad de usuarios con información geográfica disponible con el fin de obtener todos sus tuits.
Los usuarios se buscaron por provincia de modo tal que haya una cantidad aproximadamente equitativa de cada una.
La búsqueda de los usuarios se hizo de la siguiente manera:

Por cada provincia de la Argentina, se extrajo las ubicaciones de cada uno de sus departamentos, de los partidos de la provincia de Buenos Aires y de las comunas de la Ciudad Autónoma de Buenos Aires. El conjunto de estas forman la subdivisión de segundo orden de la república Argentina. La lista de departamentos/partidos/comunas fue extraída a partir de los datos publicados del Censo Argentino del año 2010. Para extraer los tuits se utilizó la librería de \textit{python} llamada \textit{tweepy}.
De esta manera se recolectó aproximadamente 2000 usuarios por provincia, lo que resulta en 46000 usuarios argentinos. Sobre este conjunto de usuarios se buscaron los tuits. Se decidió no tener en cuenta los retuits dado que no son escritos por los usuarios sino que son una mera copia de otros tuits. 


\subsection{Búsquedas geolocalizadas}

Las búsqueda geolocalizada es una herramienta provista por la API (\textit{Application Programming Interface}) de \textit{Twitter} en la cual primero se intenta de buscar tuits cuyas coordenadas sean las buscadas. En caso de no tener éxito, se busca aquellos tuits creados por usuarios que tienen en el campo \textit{location} de su perfil un lugar cuyo geocódigo coincida con el de sus coordenadas. Es decir, si se hace una búsqueda inversa de las coordenadas, devuelve el lugar de su perfil.

Una vez obtenida la lista de ubicaciones, se realizaron búsquedas por cada provincia con centro en las coordenadas de los departamentos de la misma y con un radio de 20 millas. Sobre el resultado de esta búsqueda, únicamente se seleccionaron los usuarios que tienen como campo \textit{location} al menos uno de los nombres de las ciudades de la provincia. Con esta precaución eliminamos los posibles tuits de turistas que escribieron en un lugar pero que no viven allí.

En el gráfico de la figura \ref{fig:busqueda_usuarios} se muestran las ubicaciones de los usuarios.


\begin{figure}[!ht]\centering
  \begin{minipage}[t]{0.31\textwidth}
    \includegraphics[width=\linewidth]{./images/mapaprovincias.pdf}
    \caption{} 
    \label{fig:mapaProvincias} 
   \end{minipage}
   \begin{minipage}[t]{0.31\textwidth}
    \includegraphics[width=\linewidth]{./images/mapadepartamentos.pdf}
    \caption{} 
    \label{fig:mapaDepartamentos} 
   \end{minipage}
   \begin{minipage}[t]{0.31\textwidth}
    \includegraphics[width=\linewidth]{./images/mapaprovinciasConPuntos.pdf}
    \caption{} 
    \label{fig:mapaPuntos} 
   \end{minipage}
   



   \caption{Ubicaciones de los usuarios: En la figura \ref{fig:mapaProvincias} se muestra un mapa de Argentina con la distribución de los usuarios en las provincias sobre el conjunto de desarrollo. El mapa de la figura \ref{fig:mapaDepartamentos} permite visualizar la distribución de los usuarios en los departamentos sobre el conjunto de desarrollo. Se muestra con áreas grises los departamentos que no poseen usuarios que hayan definido el campo ubicación de su perfil en ese lugar. El mapa \ref{fig:mapaPuntos} muestra la distribución de las coordenadas obtenidas a partir de todos los usuarios del conjunto de desarrollo. Las coordenadas fueron obtenidas a través de un proceso de geocodificación. } 
  \label{fig:busqueda_usuarios} 

\end{figure}

En la figura \ref{fig:mapaProvincias} se ve que la distribución de usuarios en las provincias es bastante pareja. Si bien en la figura \ref{fig:mapaDepartamentos} hay regiones grises que indican la ausencia de usuarios en ese lugar, cabe destacar que los mapas se realizaron obteniendo las coordenadas geográficas a partir de la ubicación definida en el perfil del usuario. Por lo tanto, si una persona declara que vive en \textit{Tucumán, Argentina}, contabilizamos como que esa persona vive en la capital de esa provincia, lo cual puede no ser cierto. Sin embargo, esto no invalida los resultados puesto que la granularidad del análisis es a nivel provincial. Finalmente para ver la distribución de las coordenadas de los usuarios a lo largo del país mostramos la figura \ref{fig:mapaPuntos}. Se puede observar que en la mayoría de las grandes ciudades hay usuarios en nuestro conjunto de datos. En el apéndice se puede encontrar un mapa \ref{fig:mapaGPS} donde se contabilizaron todos los tuits con coordenadas geográficas del conjunto de datos. En este gráfico se puede observar que la distribución es mucho más amplia, aunque sigue habiendo más concentración de usuarios en aquellos departamentos con más densidad poblacional.

Si bien en este trabajo nos enfocamos en las coordenadas de las localidades dentro de Argentina, basta con cambiar las coordenadas y los nombres de las localidades que tienen que tener los campos \textit{location} para realizar un análisis sobre otros países en una segunda etapa.

% \section{Datos de desarrollo y de validación}

% Por cada provincia se tomó a los usuarios de la misma y se los dividió para tener un conjunto de datos de desarrollo y uno de validación. El conjunto de validación fue creado para poder corroborar que los resultados obtenidos por el análisis del conjunto de desarrollo no sean algo intrínseco de esta muestra, sino que se pueden extrapolar a toda la población. 
% La división de los datos se realizó de manera tal que los conjuntos resultantes sean lo más independientes posibles: 
% \begin{description}
%     \item [Usuarios disjuntos] Debido a que ciertos usuarios repiten palabras constantemente, ya sea porque son bots o simplemente porque hablan siempre de los mismos temas, es adecuado validar los resultados con textos producidos por distintos usuarios. De esta manera se intentó mitigar el ruido generado por estos usuarios particulares.
%     \item [Fechas disjuntas] Al analizar los resultados sobre los textos generados en un tiempo acotado de tiempo, estamos trabajando con una muestra específica que es de esperar que tenga fenómenos particulares debido al momento en que fueron escritos. Por ejemplo, debido a cierto fenómeno climático o en el transcurso de un evento polémico (como un debate presidencial o un torneo deportivo), se pueden obtener tuits con una frecuencia de ciertas palabras muy distinta a la frecuencia que tiene normalmente. Por esta razón se dividieron los tuits producidos por los usuarios de manera tal que sus fechas sean disjuntas.
% \end{description}

% La división fue de la siguiente manera:
% Sobre el conjunto de usuarios se dividió en dos de forma aleatoria, con lo que se obtuvo $Usuarios_1$ y $Usuarios_2$. Luego se buscó la fecha $Fecha_{DIV}$ por la cual había una cantidad equiparable entre el conjunto de tuits producidos por  $Usuarios_1$ antes de $Fecha_{DIV}$ y el conjunto de tuits producidos por $Usuarios_2$ después de $Fecha_{DIV}$. Para encontrar la fecha se hizo una búsqueda binaria: dada una fecha $Fecha_{temp}$ (inicialmente es el día intermedio entre la fecha del primer y último tuit recolectado) se calcula la cantidad de tuits que hay con fechas anteriores sobre el conjunto de usuarios $Usuarios_1$ y análogamente se calcula la cantidad de tuits posteriores a esa fecha sobre el otro conjunto de usuarios. Si la cantidad del primer conjunto de tuits es menor que la del segundo conjunto, la $Fecha_{DIV}$ se busca en  el intervalo [$Fecha_{temp}-FechaFinal$] y se repite el procedimiento. Por lo tanto se cumple la siguiente ecuación:

% \begin{equation}
% %\sum_{ f = FechaInicial}^{Fecha_{Div}} tweets(Usuarios_1,f) \approx \sum_{ f = Fecha_{Div}}^{FechaFinal} tweets(Usuarios_2,f) 
%  Fecha_{DIV}  = \operatorname*{arg\,min}_{F} \left|\sum_{ f = FechaInicial}^{F} tuits(Usuarios_1,f) - \sum_{ f = F}^{FechaFinal} tuits(Usuarios_2,f)\right|
% \end{equation}
% Donde $FechaInicial$ es la fecha del primer tuit recolectado  y $FechaFinal$ es la del último.\\

% Después de fijar la fecha se dividió el conjunto de tuits producidos por estos usuarios: el conjunto de desarrollo con los tuits producidos antes de $Fecha_{DIV}$ y el conjunto de test producidos posteriormente a esa fecha.

\section{Tokenización y normalización}

En cuanto al análisis del texto surge una primer problemática: ¿qué es una palabra? En principio podemos definir a una palabra como cualquier secuencia de caracteres delimitados por espacios blancos. Con esta definición \textit{523456} y \textit{?} serían palabras. Debido a esto podríamos restringir nuestra definición a una secuencia de caracteres alfabéticos. Los ejemplos mencionados anteriormente dejarían de estar dentro de la definición. Sin embargo términos como \textit{asdsdafsdf} también serían palabras. Para evitar este problema podríamos tener un diccionario como filtro para saber si una secuencia de caracteres dada es una palabra. Si bien esto tendría mucha precisión al momento de filtrar los términos, no sería capaz de detectar palabras que existen en una lengua pero que no están incluidos en el diccionario elegido. Además , dada la cantidad de palabras recogidas, es altamente improbable que una secuencia al azar de caracteres alfabéticos reúna las condiciones de frecuencia necesarias para resultar destacada por la métrica que utilizamos. Es por eso que decidimos tomar a una palabra como una secuencia de caracteres alfabéticos.

Es muy posible que tengamos palabras que no sean interesantes a nivel lingüístico, como errores de tipeo (e.g computadira, escribur), errores ortográficos o  nombres propios. Es importante destacar que Twitter tiene caracteres especiales para mencionar a la gente, como el @, o el \#(hashtag) utilizado para agrupar mensajes. Estos caracteres aparecen mucho, ya que los usuarios suelen responderse en la red, mencionando los mismos temas (aclarando el hashtag), o respondiendo a otros usuarios. Ya que esos caracteres no son alfabéticos, cualquier término que los utilice no va a ser parte del conjunto de palabras, como tampoco lo serán las direcciones de páginas web. Decidimos que se filtren estos términos ya que no tienen interés lingüístico y además agregarían mucho ruido a los datos.

Además de la tokenización del texto, se realizó una normalización sobre él. Todas las letras se convirtieron a letra minúscula y las palabras con más de tres letras iguales de forma consecutiva se redujeron para que solo tengan tres repeticiones. De esta forma, el término \textit{padreeeee} y \textit{padreeee} fueron reducidos a una única unidad léxica (\textit{padreee}). Esto se hizo con la librería \textit{TweetTokenizer de NLTK}. 
Se descartó la idea de filtrar las palabras que no estuvieran en un diccionario ya que si bien hubiera eliminado mucho ruido, también nos hubiera filtrado palabras de interés. Este es el caso de los neologismos, o las palabras que, si bien se utilizan hace mucho tiempo, no están en los diccionarios actuales.

\section{Caracterización de la muestra}

Para tener una noción más completa de la muestra, se presenta la tabla \ref{tab:cantidades} que indica las cantidades de palabras y tuits por provincia.


\begin{table}[ht]
\centering


\begin{tabular}[width=0.7\textwidth]{|l|c|c|c|c|c|c|}
\hline
Provincia      & \#Palabras Distintas & \#Usuarios & \#Tuits & \#Total Palabras \\ \hline
Buenos Aires   & 191919       & 920          & 1125042    & 8974372  \\
Catamarca      & 173104       & 957          & 1057019    & 8161309   \\
Chaco          & 169476       & 964          & 976943     & 7605991   \\
Chubut         & 182592       & 954          & 1023373    & 8884745   \\
Córdoba        & 207307       & 987          & 1224266    & 10075932  \\
Corrientes     & 183292       & 939          & 1044951    & 8426940   \\
Entre Ríos     & 188679       & 969          & 1193693    & 9462986  \\
Formosa        & 169254       & 903          & 923352     & 7184382   \\
Jujuy          & 171064       & 971          & 678004     & 5951778   \\
La Pampa       & 186593       & 935          & 1085757    & 8996318  \\
La Rioja       & 186041       & 946          & 704044     & 6757277  \\
Mendoza        & 193708       & 945          & 1099717    & 9402399   \\
Misiones       & 168400       & 972          & 984218     & 7790197   \\
Neuquén        & 188038       & 927          & 1111201    & 9021449   \\
Río Negro      & 194383       & 965          & 1215361    & 9991831  \\
Salta          & 188402       & 884          & 830916     & 7506652   \\
San Juan       & 183546       & 926          & 1002322    & 8377792  \\
San Luis       & 164185       & 896          & 1006464    & 8327093  \\
Santa Cruz     & 174089       & 935          & 876621     & 7432923  \\
Santa Fe       & 201879       & 937          & 1019620    & 8862328  \\
Santiago del Estero       & 166540       & 887          & 944109     & 7355729  \\
Tierra del Fuego & 197273       & 964          & 976426     & 8559218   \\
Tucumán        & 195643       & 962          & 1093874    & 9238526 \\
  \hline
\end{tabular}
\caption{Cantidades del conjunto de datos}
\label{tab:cantidades}
\end{table}



También analizamos la cantidad de palabras por tuit promediadas sobre cada usuario.
Debido a que los tuits están limitados a 140 caracteres, era de esperar que no hubiera demasiadas palabras promedio por cada tuit. En la figura \ref{fig:cantPalabrasUsuario} podemos observar que la media para la cantidad de palabras promedio en un tuit está entre 7 y 8.
\begin{figure}[!ht]\centering
  \begin{minipage}[t]{0.49\textwidth}
    \includegraphics[width=\linewidth]{./images/train/conFiltro/cantPalabrasUsuario.pdf}
    \caption{Histograma de la cantidad de palabras totales por cada usuario.} 
    \label{fig:cantPalabrasUsuario} 
   \end{minipage}
   \begin{minipage}[t]{0.49\textwidth}
    \includegraphics[width=\linewidth]{./images/train/conFiltro/cantPalabrasPromedio.pdf}
    \caption{Histograma de la cantidad de palabras promedio para todos los usuarios.} 
    \label{fig:cantPalabrasPromedio} 
   \end{minipage}
   
\end{figure}

%Cantidad de palabras promedio por tweet
%Cantidad de palabras por usuario + varianza por provincia
\subsection{Distribución temporal de tuits}
Los tuits recolectados para el conjunto de datos de desarrollo tienen una particularidad: a medida que pasan los años 
hubo mayor cantidad de tuits durante un año. Esto se refleja en los gráficos de las figuras \ref{fig:histTweetsProvincia1} y \ref{fig:histTweetsProvincia2}.

\begin{figure}[!ht]\centering
   \begin{minipage}[t]{0.49\textwidth}
     \includegraphics[width=\linewidth]{./images/train/sinFiltro/histTweetsProvincia1_sinFiltro.pdf}
     \caption{}
     \label{fig:histTweetsProvincia1}
   \end{minipage}
   \begin {minipage}[t]{0.49\textwidth}
     \includegraphics[width=\linewidth]{./images/train/sinFiltro/histTweetsProvincia2_sinFiltro.pdf}
     \caption{}
     \label{fig:histTweetsProvincia2}
   \end{minipage}
   \caption { En la figura \ref{fig:histTweetsProvincia1} se presenta un histograma donde se muestra la cantidad de tuits que se hicieron por intervalo de tiempo en la provincias La Pampa y Buenos Aires. En la figura \ref{fig:histTweetsProvincia2}, se presenta el gráfico para Chaco y Neuquén.}
\end{figure}


% Tweets promedio por usuario, por provincia. (varianza)

% cantidad de tweets por usuario, cantidad media de palabras por usuario

\chapter{Métricas para detectar palabras salientes}
\label{ch:metricas}
\section{Búsqueda de contrastes}

% Definir que es un contraste, que tiene un uso significativo en una region más que en otra. LAs alternativas que propusimos, z test binomial.
Una palabra tiene un contraste cuando esta tiene un uso con diferencias significativas en
distintas regiones. En este trabajo nos propusimos crear un listado con palabras con contrastes que tengan
importancia a nivel lingüístico. En este sentido, los nombres de personas, lugares u organizaciones no 
fueron considerados de interés a pesar de tener contrastes en su uso.
Este listado fue ordenado por una métrica que capte en un único valor el nivel contrastivo. De esta manera, 
se seleccionó un subconjunto de palabras, de acuerdo a la métrica, el cual fue analizado manualmente en otros textos por la Academia Argentina de Letras.

El primer acercamiento para ver el contraste de las palabras lo realizamos comparando las frecuencias de las palabras 
en cada par de provincias de la Argentina. Para esto calculamos, por cada palabra, la frecuencia de ocurrencias sobre cada una de las dos provincias. La mayor frecuencia de ambas, la llamamos frecuencia máxima y a la menor, la frecuencia mínima. Luego el cociente entre la frecuencia máxima y la frecuencia mínima tiene como resultado lo que llamamos \textit{maxDif}. En caso de que en una de las dos provincias no se haya 
recolectado tuits con esa palabra, se tomaba como frecuencia mínima a la frecuencia mínima distinta de 0 de todas las palabras generadas en esa provincia. Así se evitó la división por cero. Esta métrica se resumen en la ecuación \ref{eq:maxDif}.


\begin{equation}
  \label{eq:maxDif} 
  maxDif(w,p_1,p_2) = \frac{F_{max}(w,p_1,p_2)}{F_{Min}(w,p_1,p_2)}
\end{equation}
donde 
\begin{equation}
F_{max}(w) = \max(frec(w,p_1),frec(w,p_2))
\end{equation}

%TODO: arreglar el overfull
\begin{equation}
 F_{min}(w) = \left\{ \begin{array}{ll}
             \min(frec(w,p_1),frec(w,p_2))  \text{ si } frec(w,p_1) * frec(w,p_2) > 0  & \\
             \\
             \min(frec(w,p))  \forall w \in palabras(p) , \text{ con } p=\{p_1,p_2\} \setminus \{P_{max}\} \text{sino} &  \\
             \end{array}
   \right.
\end{equation}
   donde $P_{max}$ es la provincia que tiene la mayor frecuencia de ambas.\\



De esa manera se ordenó el listado de cada par de provincias teniendo en cuenta la división de frecuencias. 
Sin embargo, este método imposibilitaba el trabajo manual para la Academia Argentina de Letras que debía mirar estos listados y hacer un análisis más exhaustivo sobre las palabras con mayor diferencia de frecuencias, debido a que había $\binom{23}{2} = 253$
listados (o equivalentemente 253 columnas en un mismo listado) a analizar. Además la métrica solo permitía saber si había un contraste entre dos provincias, pero no se podía tener en cuenta la frecuencia de la palabra en el resto de las provincias. 
% Mencionar que la idea era realizar un z test para obtener las palabras más significativas.
En consecuencia las palabras se encontraban repetidas en los distintos listados y con diferentes valores de \textit{maxDif}, lo cual hacía muy difícil poder identificar en que regiones había una diferencia significativa de frecuencias.

Debido a esto decidimos realizar un nuevo enfoque para encontrar las palabras con alta contrastividad en las distintas regiones, de manera que una métrica pueda reflejar el nivel de contrastividad de la palabra en un único valor.
De este modo, nos enfocamos en analizar el contraste de frecuencias de palabras sobre las provincias a través de una métrica superadora.

\subsection{Métricas para medir el contraste en la frecuencia de las palabras}
Dado que se quieren encontrar las palabras con contrastes significativos en distintas 
regiones se propone generar una métrica basada en la cantidad de información 
para poder realizar esta tarea.

Una medida que se puede usar para comparar las frecuencias de las palabras en las diferentes regiones del país puede ser la entropía definida por Shannon ( ver en el apéndice: \ref{sub:entropiaShannon}), debido a que podemos tener un valor que informe qué tan uniforme es la distribución de las frecuencias de cada palabra.
Sin embargo, la entropía como única medida tiene sus desventajas. Principalmente, una palabra con una sola ocurrencia en una provincia y ninguna en las demás, tiene la entropía mínima. A pesar de que nos interesan las palabras con un contraste significativo entre regiones, dentro de ellas elegiremos las que tienen mayor cantidad de ocurrencias. Es por esto que elaboramos otra métrica que tenga en cuenta la entropía, entre otras variables a tener en cuenta.


\subsection{Valor de información}
La métrica que utilizamos para ordenar los listados de palabras y detectar cuáles son
las que tienen altos contrastes en su uso en distintas regiones fue inspirada por el
trabajo de Zanette y Montemurro \cite{montemurro2010towards}.
Ellos, a diferencia de Shannon, estudiaron una relación entre una medida de la información y su función semántica en el lenguaje.
A continuación detallamos el procedimiento para calcular lo que ellos llamaron
el valor de la información:

Dado un texto dividido en P partes iguales, se calcula la entropía  $H(w)$ sobre el vector de cantidad de ocurrencias en cada una de las P ventanas.
Luego se define $\widehat{H(w)}$  como la entropía de una permutación aleatoria del texto y promediada por todos las posibles realizaciones de la permutación de él. 
%% CHEQUEAR LA DEFINICION DE LA ENTROPIA SHUFFLEADA

Es decir, se distribuyen uniformemente las palabras en P partes y se calcula la
entropía como se hizo con el texto original. Es de esperar que en la mayoría de casos 
la entropía del texto permutado sea mayor que la medida en el calculo original. Esto 
se debe a que las palabras se distribuyen de forma más uniforme 
en las distintas partes.

Finalmente, definen al valor de la información como $I(w) = p(w) (\widehat{\eta(w)} - \eta(w))$, con $p(w)$ la frecuencia total de la palabra en el texto. 
De esta manera se les da más importancia a las palabras que son más frecuentes y a las palabras que tienen una baja entropía, ya que en estas el término de la diferencia es más grande.

Este estudio se hizo sobre tres textos, \textit{Análisis de la mente}, 
\textit{Moby Dick} y \textit{El origen de las especies} de Charles Darwin. 
En los tres libros las palabras con mayor valor de la información están 
altamente relacionadas con los temas principales.

Si bien esta métrica tiene en cuenta la frecuencia de las palabras además de la 
entropía, el texto en Twitter resulta difícil de dividir en partes iguales. 
Esto es porque la división está pensada para dividir el texto en secciones que 
posiblemente hablen de distintos temas y nuestros textos son tuits que por lo general no superan las 10 palabras.
Otra dificultad que surge de esta métrica es la imposibilidad de realizar la media 
de todas las posibles permutaciones del texto por la limitación computacional ya que 
tenemos una cantidad muy grande de datos.Es por eso que realizamos una métrica parecida.

Podemos pensar a las palabras del texto como una variable aleatoria W, donde cada palabra w tiene una probabilidad de aparición en una provincia dada de la Argentina. Esta probabilidad la aproximamos con la frecuencia en la que aparece, es decir la cantidad de ocurrencias de la palabra dividida por la cantidad de palabras totales.
Por otro lado sea P una variable aleatoria que cuenta la cantidad de personas que 
utilizan la palabra p en cada provincia.

Luego,
\begin{equation}
%I(w) =  norm_{p}(w) * norm_{u}(w) * (\widehat{H}_{w}(w) - H_{w}(w)) * (\widehat{H}_{p}(w) - H_{p}(w)) \\
I(w) =  I_p(w) * I_u(w)\\
\end{equation}
\begin{equation}
I_p(w) = norm_{p}(w) * (\widehat{H}_{w}(w) - H_{w}(w)) \\
\end{equation}
\begin{equation}
I_u(w) = norm_{u}(w) * (\widehat{H}_{u}(u) - H_{u}(w))
\end{equation}

\begin{equation}
norm_{p}(p) = \frac{cw(p)- MIN_W }{MAX_W - MIN_W}
\label{eq:norm1}
\end{equation}

\noindent\begin{minipage}{.5\linewidth}
\begin{equation}
\text{donde:}  MIN_W = \min\limits_{p \in Palabras} cw(w)
\end{equation}
\end{minipage}%
\begin{minipage}{.5\linewidth}
\begin{equation}
  MAX_W = \max\limits_{p \in Palabras} cw(w)
\end{equation}
\end{minipage}
y $cw(p)$ es igual al logaritmo sobre la cantidad de ocurrencias de esa palabra en toda la Argentina, es decir $cw(p) = \log_2(cantidadOcurrencias(p))$.

Análogamente,
\begin{equation}
norm_{u}(w) = \frac{cu(w)- MIN_U }{MAX_U - MIN_U}
\label{eq:norm2}
\end{equation}
\noindent\begin{minipage}{.5\linewidth}
\begin{equation}
 \text{donde: } MIN_U = \min\limits_{p \in Palabras} cu(p)
\end{equation}
\end{minipage}%
\begin{minipage}{.5\linewidth}
\begin{equation}
  MAX_U = \max\limits_{p \in Palabras} cu(p)
\end{equation}
\end{minipage}
y $cu(p)$ es el logaritmo sobre  la cantidad de usuarios que utilizan dicha palabra en la Argentina, es decir $cu(p)= \log_2(cantidadUsuarios(p)))$.
$\widehat{H}$ es la entropía con las cantidades distribuidas uniformemente y H es la entropía común.

Tanto $norm_{w}$ como $norm_{u}$ realizan una normalización del logaritmo de esas variables. Esto se debe a que el logaritmo genera una dispersión en las medidas de forma tal que su distribución sea más uniforme a lo largo de todo el rango de valores. Esto se puede ver en la figura \ref{fig:cantNorms}.
$\widehat{H}_{u}$ y $\widehat{H}_{w}$ se corresponde a las entropías de los vectores simulados de apariciones. %TODO: explicar más en detalle
Esta simulación se realiza con una distribución multinomial ya que se distribuye la suma de los valores de la variable aleatoria uniformemente. 

% La variacion de la entropia se puede ver como una cantidad que mide cuanta información se necesita para poder obtener esa distribución
% Si las cantidades de ocurrencias (o de personas que utilizan) de un término están distribuidas uniformemente a través de todas las provincias, quiere decir que no aporta demasiada información 

\begin{figure}[!ht]
\centering
\includegraphics[width=1.0\textwidth]{./images/train/sinFiltro/cantNorms_sinFiltro.pdf}
\caption{Cantidades y sus normalizaciones: Mediante los histogramas presentamos la distribución de los valores de la cantidad de los usuarios totales que utilizaron cada palabra, como la cantidad de ocurrencias de una palabra. A la derecha se pueden observar la normalización de las variables descriptas en \ref{eq:norm1} y \ref{eq:norm2}.} 
\label{fig:cantNorms} 
\end{figure}

El término de la diferencia de la entropía sobre la cantidad de personas que utilizan la palabra tiene como objetivo mitigar el ruido de la entropía de palabras. En particular una determinada provincia o región pueden tener muchas ocurrencias de una palabra causado por algunos usuarios que utilizan constantemente el término. Un ejemplo de esto podrían ser bots que escriben automáticamente textos iguales (o similares) en grandes cantidades. Otra posible causa de este fenómeno podría ser la de usuarios de ciertas organizaciones que hablan de personas, lugares o marcas de forma constante.

Para eliminar los valores atípicos se procedió a remover las palabras que no superaran las 40 ocurrencias, como también aquellas que eran dichas por menos de 6 usuarios. 

\subsection{Frecuencia de las palabras}
\label{sub: frecuenciaPalabras}
% Ver si se hace este gráfico con todas las palabras ya que por ahora tengo el conjunto de palabras con más de 40 ocurrencias
En la figura \ref{fig:cantPalabras} graficamos la distribución de la cantidad de ocurrencias de las palabras.Podemos observar que la mayoría de las palabras ocurren poco. En particular el 50\% de las palabras ocurren menos de 139 veces. Por otro lado hay pocas palabras que ocurren mucho, por ejemplo la palabra \textit{que} o la preposición \textit{de}.

\begin{figure}[ht]
\centering
\includegraphics[width=0.8\textwidth]{./images/DistribucionOcurrenciasPalabrasAnotaciones.pdf} 
\caption{Histograma de la cantidad de ocurrencias de las palabras} 
\label{fig:cantPalabras} 
\end{figure}

\begin{table}[ht]
\centering
\label{tab:palabrasMasOcurrentes}
\begin{tabular}{|c|c|}
\hline
Palabra & Cantidad de Ocurrencias \\ \hline
que     & 7509160                 \\
de      & 6527014                 \\
a       & 4962492                 \\
la      & 4913854                 \\
no      & 4177810                 \\
me      & 4101998                 \\
y       & 3838370                 \\
el      & 3773455                 \\
en      & 2969783                 \\
te      & 2060662                 \\
se      & 1976027                 \\
un      & 1863075                 \\
es      & 1825892                 \\
con     & 1799979                 \\
lo      & 1712189                 \\
mi      & 1643777                 \\
por     & 1553382                 \\
los     & 1498941                 \\
para    & 1398757                 \\
las     & 1212452                 \\
\hline
\end{tabular}
\caption{Cantidad de apariciones de las 20 palabras más frecuentes.}

\end{table}

Si comparamos la posición de la palabra en un listado ordenado podemos ver que se cumple con la ley de Zipf. Esta es una ley empírica formulada por George Zipf en el año 1932 en la cual se establece una relación entre la frecuencia de una palabra con su posición dentro del listado de palabras ordenadas por frecuencia decreciente. En particular, sea $n$ la posición de la palabra en el listado ordenado y sea $f(n)$ la cantidad de ocurrencias de la n-esima palabra, se puede hacer la siguiente aproximación:

$$f(n) \approx \frac{1}{n^{\alpha}}$$
donde $\alpha$ toma un valor levemente mayor a 1.
Entonces, bajo la ley de Zipf uno puede saber que la frecuencia de la segunda palabra más dicha en un corpus, es aproximadamente la mitad que la primera. La palabra con posición 3 en el listado ordenado por frecuencias, va a tener aproximadamente la tercera parte de la cantidad de ocurrencias que la primera y así sucesivamente. Otra forma de utilizar esta ley empírica es la siguiente:
sabiendo la posición de una palabra \textit{w} en el listado ordenado por frecuencias de un corpus \textbf{A} y sabiendo la cantidad de palabras totales de un corpus \textbf{B}, puede estimarse la cantidad de ocurrencias de \textit{w} en el corpus \textbf{B}.
%TODO: HACER EJEMPLO de calculo de frecuencias a traves de la frecuencia de nuestro corpus

\begin{figure}[!ht]
\centering
\includegraphics[width=0.5\textwidth]{./images/zipf.pdf}
\caption{Cantidad de Ocurrencias de palabra vs posición en listado ordenado. Se aplicó el logaritmo a las cantidades de ocurrencias, como también a los valores de las posiciones para mostrar la proporcionalidad entre $f(s)$ y $\frac{1}{s^{\alpha}}$.} 
\label{fig:zipf} 
\end{figure}



%Qué resultados obtuvimos del análisis en cuestión. Acá vamos a poner qué palabras fueron significativas, dónde lo fueron, y demás.

% Distribucion de entropía vs frecuencia  
% Distribucion de frecuencias de las palabras
% Distr entropia
% de valor de la inf. a partir de los 5000 se ve que se estanca (graficar hasta 10**5)

% subsection de la region de las palabras --> muestro tabla
% se muestra que las regiones son contiguas geograficamente (hablarlo con santiago)

% Las palabras que se vieron, las primeras generalmente se refieren a lugares o gentilicios



% ver los graficos que hace zanette en su paper

% se busco el dataset de localidades para filtrar las palabras que son lugares

\section{Entropía}
Teniendo el listado de palabras hicimos un cálculo de entropía tomando en cada provincia la cantidad de ocurrencias de cada palabra. En la figura \ref{fig:entropiaPalabras} podemos observar la distribución del valor de la entropía sobre todas las palabras con más de 40 ocurrencias y dichas por más de 5 usuarios.

Podemos ver que la mayor parte de las palabras tienen un valor de entropía entre 2.5 y 3. Esto quiere decir que hay un gran conjunto de palabras que tiene una cantidad de ocurrencias relativamente uniforme a lo largo de todas las provincias. Sin embargo hay otro conjunto de palabras que tienen una entropía menor a 2, la cual podemos considerar como baja. Estas últimas palabras serán las que tienen mayor interés debido a que tienen una variación marcada en cuanto a su utilización en las distintas regiones.

Sin embargo, ver solamente la entropía de las palabras nos puede generar la detección de palabras que no son de interés, ya sea porque no ocurren una cantidad significativa de veces o porque la variación de las ocurrencias en las distintas provincias se debe solamente a pocos usuarios que la utilizan mucho. Es por esto que también se calculó la entropía teniendo como variable la cantidad de personas que utilizaron cierto término en una determinada provincia.


\begin{figure}[ht]
\centering
\includegraphics[width=1.0\textwidth]{./images/DistribucionEntropia.pdf}
\caption{Histograma del valor de la entropía de las palabras ($H_w$).} 
\label{fig:entropiaPalabras} 
\end{figure}



\section{Valor de la información}
\label{sec:ValorDeLaInformacion}
En el gráfico \ref{fig:infoValue} se muestra una clara relación entre la cantidad de ocurrencias que tiene una palabra y su valor de la información, indicado por el colo: cuanto más oscuro más alto el valor. A su vez, se nota que el valor de la información suele ser mayor a medida que el valor de la entropía es menor. Esto no siempre es el caso debido a que hay palabras que tiene una entropía de palabras baja, pero sin embargo la entropía de personas es alta logrando que el valor de la información sea bajo.

\begin{figure}[ht]
\centering
\includegraphics[width=1.0\textwidth]{./images/entropiaPersonasxNormCantPersonas.pdf}
\caption{Gráfico de dispersión que muestra la posición en el listado ordenado según el valor de la información a partir de la escala cromática. Las posiciones más bajas aparecen más blancas. A su vez se muestra para cada palabra el valor de la entropía de las personas ($H_u$) y la cantidad normalizada de personas que utiliza dicho término($norm_p$). } 
\label{fig:infoValue} 
\end{figure}

En la figura \ref{fig:ivalue} se puede ver el valor de la información según la posición en la que se encuentra en el listado ordenado por la métrica. Notamos que el valor se estabiliza aproximadamente a partir de la palabra cuya posición es 4000 acercándose a 0.


\begin{figure}[ht]
\centering
\includegraphics[width=0.6\textwidth]{./images/train/conFiltro/valorInformacionCorte.pdf}
\caption{Distribución del valor de la información según la posición de la palabra en el listado de palabras. El gráfico se realizó sobre el conjunto de palabras cuya cantidad de ocurrencias era mayor a 40 y la cantidad de usuarios que utilizaron cada término era mayor a 5. } 
\label{fig:ivalue}
\end{figure}


\section{Proporción acumulada de ocurrencias} % (fold)
\label{proporcionDeOcurrencias}

Para tener una mejor noción sobre la métrica, medimos el porcentaje de las ocurrencias de las palabras que muestran contrastes según nuestra métrica cubiertas por subconjuntos de provincias de manera tal que para cada palabra y un número fijo de provincias, la región con esa cantidad de provincias tenga un cubrimiento máximo. 
En la figura \ref{fig:propAcum} mostramos la proporción acumulada de las palabras, tomando diferentes muestras de palabras. Es notable la diferencia de proporciones acumuladas según la muestra de palabras. Solamente con una provincia para cada palabra ya se puede cubrir, en promedio, el 76\% del total de ocurrencias sobre las mil palabras con mayor valor de la información.

%TODO: cambiar el grafico con todas las palabras candidatas 
En el gráfico \ref{fig:propAcum5000} se observa que la variación del cubrimiento de ocurrencias es menor a medida que se aumenta la cantidad de provincias. 



\begin{figure}[!ht]\centering
  \begin{minipage}{0.49\textwidth}
    \includegraphics[width=\linewidth]{./images/PropAcum.pdf}
    \caption{Proporción de ocurrencias acumulada según la muestra de palabras.} 
    \label{fig:propAcum} 
   \end{minipage}
   \begin{minipage}{0.49\textwidth}
    \includegraphics[width=\linewidth]{./images/PropAcum5000.pdf}
    \caption{Variación de la proporción de ocurrencias acumulada a partir de la muestra con las primeras 5000 palabras con mayor valor de la información.} 
    \label{fig:propAcum5000} 
   \end{minipage}
   
\end{figure}

% \begin{figure}[h]
% \centering
% \includegraphics[width=0.5\textwidth]{./images/PropAcum5000.pdf}
% \caption{Proporciones Acumuladas} 
% \label{fig:propAcum5000} 
% \end{figure}




\chapter{Análisis de las palabras contrastivas encontradas}
\label{ch:validacion}
\section{Palabras candidatas} % (fold)
\label{palabras_candidatas}
Para buscar las palabras candidatas a tener contrastes significativos en cuanto a la cantidad de ocurrencias en distintas provincias, elegimos el conjunto de las primeras 
cinco mil (5000) palabras con mayor valor de la información. El número 5000 surgió de ver la distribución de los valores de la información graficado en la figura \ref{fig:ivalue}, donde hay una caída pronunciada de la métrica y a partir de la palabra cuya posición es 4000 se observa que empieza a estabilizarse con valores muy cercanos a 0. Es por esto que nos pareció razonable dar un margen de 5000 palabras para evaluar manualmente las palabras del listado y, entre estas, seleccionar las palabras con contrastes significativos que tienen interés a nivel lingüístico.

Como era de esperar los topónimos, como los nombres de ciudades y provincias, son palabras que ocurren mayormente en sus respectivas regiones. Esto causa que haya una gran variación en 
la cantidad de ocurrencias sobre las distintas provincias, lo que genera un valor alto en la métrica de valor de la información. Para facilitar la detección de palabras contrastivas con 
mayor interés lingüístico buscamos un conjunto de datos con los nombres de las localidades y departamentos de la República Argentina de modo tal que podamos resaltarlas para que el equipo de filólogos tenga una primera alerta sobre posible toponimia.

% Agregar algún comentario sobre regiones dialectales conocidas


\section{Regiones de palabras} % (fold)
\label{sub:regiones_de_palabras}

Una vez que calculamos las regiones que cubren un umbral para cada palabra, nos propusimos analizar cuales son las más frecuentes. Para eso generamos una lista con los conjuntos de provincias cuya cantidad sea menor a 7 y los ordenamos según su frecuencia. En la tabla \ref{tab:regiones} se muestran los primeros conjuntos de provincias obtenidos a partir de las primeras 5000 palabras con mayores contrastes de acuerdo al valor de la información. Más precisamente se muestran las 30 regiones con mayor cantidad de palabras y cuya cantidad de provincias sea mayor a uno. Analizando esta tabla pudimos notar que la mayoría de las regiones están compuestas por provincias contiguas, es decir que para cada provincia dentro de una región hay otra provincia limítrofe. Mostramos algunos de los conjuntos de palabras de cada región en la tabla \ref{tab:palabrasRegiones} (ver apéndice). 

\begin{table}
\centering

\begin{tabular}{|c|c|}
\hline
Conjunto de provincias                                 & Cantidad de Palabras  \\ \hline
Jujuy - Salta                                          & 24          \\
Mendoza - San Juan                                    & 19          \\
Neuquén - Río Negro                                   & 18          \\
Corrientes - Misiones                                 & 16          \\
Chaco - Corrientes - Formosa                           & 16          \\
Chaco - Corrientes                                    & 16          \\
Chubut - Santa Cruz                                   & 13          \\
Catamarca - La Rioja                                  & 12          \\
Santa Cruz - Tierra del Fuego                         & 12          \\
Corrientes - Entre Ríos - Formosa - La Rioja - Misiones  & 12          \\  %La Rioja está de más
Formosa - Misiones                                    & 12          \\
Corrientes - Formosa - Misiones                        & 12          \\
Córdoba - La Rioja                                    & 11          \\
Catamarca - Salta - Santiago del Estero - Tucumán       & 11          \\
Catamarca - Jujuy - La Rioja - Salta - Santiago del Estero - Tucumán & 10          \\
Chaco - Corrientes - Misiones                          & 10          \\
Chaco - Corrientes - Formosa - Misiones                 & 9           \\
Catamarca - Santiago del Estero - Tucumán              & 9           \\
Catamarca - Tucumán                                   & 9           \\
Salta - Tucumán                                       & 8           \\
Catamarca - Jujuy - Salta - Santiago del Estero - Tucumán& 8           \\
Neuquén - San Juan                                    & 7           \\ %no son contiguas pero son cercanas
Chubut - Santa Cruz - Tierra del Fuego                 & 7           \\
Buenos Aires - La Pampa                                & 7           \\
Salta - Santiago del Estero - Tucumán                  & 7           \\
Buenos Aires - La Pampa - Río Negro                     & 6           \\
Corrientes - Formosa                                  & 6           \\
Catamarca - Jujuy - Salta - Tucumán                     & 6           \\
Chaco - Corrientes - Entre Ríos - Formosa - Misiones     & 6           \\
Catamarca - Santiago del Estero                       & 6           \\
\hline
\end{tabular}
\caption{Indica cuantas palabras tienen un cubrimiento del 80\% de sus ocurrencias en cada conjunto de provincias a partir de las 5000 palabras con mayor contrastes (de acuerdo al valor de la información).}
\label{tab:regiones}
\end{table}


\section{Problemas en el conjunto de datos}
\label{problemas_datos}

Cuando vimos las palabras con mayor valor de la información, observamos que algunas palabras de la provincia de La Rioja eran provenientes de España. Analizando la causa de este problema, notamos que la API de Twitter no realiza las búsquedas localizadas como uno esperaría. En particular, no solo se fija en los tuits geolocalizados, sino que también hace una búsqueda inversa a través de los nombres de las ciudades que tienen esa coordenada. Específicamente, La Rioja es una provincia Argentina, como así también una provincia de España. Es por eso que al hacer búsquedas con las coordenadas de ciudades de La Rioja en Argentina, tuvimos resultados de tuits de España. Lo mismo sucedió con San Juan (capital de Puerto Rico), Santiago Del Estero (Santiago de Chile) y Córdoba (ciudad de Andalucía, España). A pesar de que los tuits no fueron escritos en Argentina, consideramos que su cantidad no es lo suficientemente grande como para tener resultados incorrectos.

% Histograma de la distribucion del valor de la información

\section{Caracterización de las palabras identificadas como contrastivas}
\label{caracterizacion_resultados}

Dentro de las palabras contrastivas identificadas a través de la métrica, podemos hacer una caracterización de ellas según el fenómeno lingüístico representan.

%A continuación presentamos en detalle cada fenómeno lingüístico y ejemplificamos con las palabras encontradas:

En base al listado de palabras identificados como contrastivas a partir de la métrica, se realizó una validación lingüística por lexicógrafos de la Academia Argentina de Letras. En esta, se realizó un estudio pormenorizado, palabra por palabra, en el cual los criterios seguidos para que una palabra sea relevante privilegiaron las posibilidades de que aquella forme parte del repertorio léxico de una comunidad de hablantes. Esto excluyó, como es tradicional en lingüística, nombres propios y topónimos locales, que la métrica sube a los puestos altos de las listas porque efectivamente su uso es abundante y contrastivo. 

En la lista \ref{it:caracterizacionLinguistica} presentamos una caracterización de las palabras. La lista apenas posee algunos ejemplos en cada categoría. Sin embargo sirve para ilustrar ejemplos de uso en el habla cotidiana de las palabras contrastivas identificadas. La lista completa arroja un resultado dentro del rango de las 300 palabras dignas de estudio por cada 5000 palabras, es decir, 1 palabra cada 17 aproximadamente. A pesar de que no existen otros proyectos que provean un término de comparación para evaluar el grado de éxito implicado en esta relación, no cabe ninguna duda de que, al menos en la detección de coloquialismos locales actualmente en uso, la herramienta plantea un verdadero punto de inflexión para la lexicografía contrastiva. Esta área del léxico es justamente la más elusiva, puesto que su impacto en cualquier medio impreso llega notablemente más tarde y, todavía más importante, en la mayoría de los casos no llega nunca. Se incluyeron como relevantes palabras que ya están incluidas en el Diccionario del Habla de los Argentinos \cite{academia2008diccionario}, dado que ese hecho es una confirmación adicional de la pertinencia de la ubicación que asignó la métrica.

Las formas cuyo uso se ejemplifica son las que están en negrita y solo en ellas se normalizó la tildación.

\begin{itemize}
  
  \label{it:caracterizacionLinguistica}
  
  \item \textbf{Coloquialismos o vulgarismos}

    \blockquote[Córdoba]{Perdon pero tenes que ser muy \textbf{culiado/a} para ir a mc y pedirte una ensalada}


    \blockquote[Mendoza]{Q \textbf{chombi} hacer un chiste y q la otra persona no se ría o no lo entienda}

    \blockquote[Neuquén]{Que \textbf{carnasas} poniendole rosas rojas a toda la ropa, para mi queda horrible sorry}

\item Indigenismos

    \blockquote[Formosa]{Te regalo ser \textbf{mitaí} y ir a jurar la bandera con el guardapolvo caliente ese y la corbata que te ahorca todo (Del guaraní mitaí “pequeño”)}

    \blockquote[Corrientes]{\textbf{Angá} mi negrito, esta triste (Del guaraní angá aprox. “pobre”) (Corrientes)}

    \blockquote[Tucumán]{Gracias tormenta \textbf{ura} por sonar como una pochoclera de chasquibums a las 3 de la mañana en mi ventana durante 50 minutos. (Valor despectivo. Del quechua ura “vulva, vagina”) }

\item \textbf{Gentilicios}

  \textbf{Casildense} (de Casilda), \textbf{concordiense} (de Concordia) y \textbf{obereño} (de Oberá).

\item \textbf{Voces no marcadas en registro, que aluden a una realidad local}

  \blockquote[San Juan]{Quiero a alguien que me diga vamos a comer \textbf{piadinas}, un pancho, un chori, una hamburguesa lo que sea y soy feliz}

  \blockquote[Misiones]{\textbf{Tareferos} que reclamaban asistencia interzafra en Posadas estarían preparando una protesta para hoy en la Fiesta del Inmigrante en Oberá.}

  \blockquote[Jujuy]{Me encantan los bohemios anti sistema que usan vans. Es como que seas ecologista y uses un cuaderno hecho con media \textbf{yunga}.}

\item \textbf{Voces sinónimas de otras más usuales en Buenos Aires}

  \blockquote[Chaco]{Teres, \textbf{pororós} y pelis con Carlita y Flor}

  \blockquote[San Juan]{Ver un negro \textbf{chuño} con musculosa y gorro.. se ve que el tipo no quería pasar ni frío ni calor.}

  \blockquote[Formosa]{Tenía la re expectativa para este sábado y al final \textbf{trancó} todo }

\item \textbf{Leísmo}

  \blockquote[Misiones]{No te olvides de \textbf{saludarle} a tu suegro hoy}

  \blockquote[Misiones]{Vine a \textbf{visitarle} a mis primas y estan re colgadas, para eso me quedaba en mi casa no maaa }

  \blockquote[Formosa]{A \textbf{esperarle} a nahuel, que traiga los teresss }

\item \textbf{Fusiones y acrónimos que pueden señalar pronunciación o alta frecuencia de uso}

  \blockquote[Buenos Aires]{Los sueños de la siesta me dejan \textbf{patra} }

  \blockquote[Córdoba]{Si mañana me dice q no, voy sola, necesito ver esa pelicula en el cine siosi}

\item \textbf{Voces consideradas generales pero que, al aparecer en la lista, permitieron verificar su contrastividad en frecuencia de uso al menos con respecto a España}

Ejemplos: \textbf{pavada}, \textbf{distrital} y \textbf{cariño}.

\item \textbf{Voces sospechadas generales pero con acepción local diferente}

  \blockquote[Mendoza]{Mañana que alguien \textbf{atine} con parque y porrones}

  \blockquote[San Juan]{\textbf{Mansas} ganas de sentarme a tomar un te con semitas}

  \blockquote[Tierra del Fuego]{\textbf{Habilítenme} una nueva espaldaa}

  \blockquote[San Juan]{sigo \textbf{asada} por cosas que han pasado hace como dos dias, que falla (Mendoza) / Que \textbf{asada} estoy, tengo la cabeza echa un lío}


\item \textbf{Voces con una morfología propia de una región}

Ejemplo: terminación aso/asa con base adjetiva.

  \blockquote[San Juan]{Creo que va a estar \textbf{malaso} lo de esta noche } 

  \blockquote[San Luis]{estoy subiendo un mix re \textbf{chomaso} que hice anoche }

  \blockquote[Córdoba]{Esta \textbf{locasa} esa mina para hacer eso}

\item \textbf{Formas indicadoras de pronunciación usual}

  \blockquote[Tucumán]{Menos mal que soy de los chetos de la carne y mañana tengo \textbf{asao} todo el dia jajajajaj}

  \blockquote[Catamarca]{Un lunes con buen humor ta \textbf{pasao} }

  \blockquote[Corrientes]{Ahora a la mañana tengo q ir hacerme la tarjebus jajajajj \textbf{mavale} q me estoy por levantarrr jajajaj}

\item \textbf{Formas verbales coloquiales con sustantivos o adjetivos como base}

  \blockquote[Neuquén]{Me calma mucho \textbf{mimosear} a mi perro }

  \blockquote[Buenos Aires]{Me vine a acostar y ya me dicen que parezco de 80 años ME CHUPA UN HUEVO LO QUE PIENSEN, DEJENME \textbf{ABUELEAR} }

  \blockquote[Tierra del Fuego]{Estaría bueno que ari venga aunque sea a saludarme y que no se quede todo el tiempo \textbf{pollereando}.}

\item \textbf{Variantes ortográficas, operativas para incorporarlas algunas como tales y también para verificar la alta frecuencia de uso}

Ejemplos: culiado (adj. despect. o fórmula de tratamiento de confianza) y tereré.

  \blockquote[Córdoba]{Q paja volver al colegio \textbf{culiaa}}

  \blockquote[Córdoba]{Que pajero el \textbf{qliao} este.}

  \blockquote[Córdoba]{Quiero recitaaal \textbf{qliaaaa}}

  \blockquote[Entre Ríos]{\textbf{Tereresss} y pile con todos mis primisss}

  \blockquote[Corrientes]{No se si hacerme un \textbf{tere} o un mate para pasar la siesta}

  \blockquote[Chaco]{Es lo mas lindo no ir al colegio y quedarme a tomar \textbf{teresss}}


\item \textbf{Vesres} : Creación de palabras por inversión de sílabas que se usa jergalmente o con fines humorísticos.

  \blockquote[Corrientes]{Estoy en lo de villa mateando con él y jimmy. Pinta \textbf{sogui} abundante más tarde dijeron }

  \blockquote[Chaco]{Uhhh me acuerdo si no habré saltado el muro del aguapey par colarme a los \textbf{cequin}. (cequín “fiesta de quince”)}

\item \textbf{Intejercciones}

  \blockquote[Formosa]{\textbf{Aijué}, encima me decís vieja, re que no pinta esto facundo jaja ya te dije como es la onda, fin }

  \blockquote[Formosa]{\textbf{Ains}, una mujer hablando de fútbol.}

  \blockquote[Corrientes]{Al fin una buena: hora libreeee! \textbf{Yirr} }

\item \textbf{Guaranismos}
\label{sub:guaranismos}
Cabe destacar la detección de términos en guaraní en la región  guaranítica\footnote{Teniendo a las regiones dialectales marcadas por Vidal de Battini}.
Un ejemplo de esto fueron las palabras \textit{angá}, \textit{angaú} y \textit{mitaí}.  Si bien estas palabras provienen del guaraní, son utilizadas en oraciones en español.
Como se puede ver en la tabla \ref{tab:guaranismos} el contraste entre las frecuencia normalizadas \footnote{La frecuencia normalizada es una medida de estandarización que indica la cantidad de veces que aparece una determinada forma por cada millón de palabras.} de la región guaranítica y la del litoral da una noción de la importancia que tienen estos términos en norte argentino. 

\end{itemize}


\begin{table}
\centering

\begin{tabular}{|l|cc|cc|}
\hline
 & \multicolumn{2}{c}{Región Guaranítica} & \multicolumn{2}{c}{Región Litoral} \\ \hline
      & \#Ocurrencias & Frecuencia Normalizada & \#Ocurrencias & Frecuencia Normalizada \\
Angá  & 548              & 45,03       & 6             & 0,21                  \\
Angaú & 205               & 16,84   & 0               & 0                     \\
Mitai & 175              & 15,69      & 1              & 0,036    \\ \hline              
\end{tabular}

\caption{Cantidad de ocurrencias y frecuencias normalizadas de las palabras en la región guaranítica y la del litoral. La cantidad total de palabras en la región guaranítica es de 12.167.635, mientras que la cantidad de términos en la región litoral es 27.477.861 }
\label{tab:guaranismos}
\end{table}

Estos términos serán agregados al diccionario del habla de los argentinos \cite{academia2008diccionario}.


\section{Validación estadística}
\subsection{Test Hipergeométrico}
%TODO: cambiar el enfoque del azar y hablar sobre si capta la nocion de contrastividad
Luego de realizar el listado de palabras ordenado por el valor de la información, se aplicó un test estadístico para tener mayor confianza de que las palabras clasificadas como contrastivas realmente tienen esta propiedad y no fueron producto del azar. Se seleccionó un conjunto de palabras significativas a nivel lingüístico a partir de las 5000 palabras consideradas más contrastivas por nuestra métrica. 

% La distribucion hipergeometrica generalmente es conocida por el modelo de la extracción con reemplazo de bolas de una urna. Hay dos colores posibles de cada bola en la urna y las bolas de un mismo color son indistinguibles. Luego, en cada extracción se puede sacar una bola exitosa En nuestro caso, la urna es el total de todas las ocurrencias de las palabras. Cada palabra se puede pensar como una bola, donde   

Decidimos elegir el test hipergeométrico ya que queremos ver que la palabra sobre la que se hace el test no estuvo sobrerrepresentada en comparación con la población. Asumimos que la cantidad de ocurrencias de una palabra se puede modelar con una distribución hipergeométrica ya que se puede pensar como un experimento donde se obtuvieron $k$ palabras exitosas en una región con $n$ palabras y un total de $N$ palabras en la Argentina. Las regiones que utilizamos para cada palabra son el conjunto de provincias que cubren el 80\% de las ocurrencias de dicho término. Luego, queremos calcular la significancia estadística de haber obtenido esas $k$ palabras exitosas.

Luego, por cada palabra seleccionada como contrastiva le aplicamos el test estadístico con la siguiente hipótesis nula: la palabra tienen un uso homogéneo en las distintas regiones de la Argentina, es decir que la frecuencia de ocurrencias de cada palabra debería ser similar independientemente de la región.
Por lo tanto, en caso de que la palabra sea contrastiva deberíamos obtener una baja probabilidad de haber obtenido diferencias entre las frecuencias de la palabra en una región con el resto del país. 
Para aplicar el test hipergeométrico representamos los datos sobre la palabra en una tabla de 2x2 como la de la tabla \ref{tab:contingencia}.
% que la cantidad de ocurrencias de la palabra en la región elegida es mayor a lo observado. 



\begin{table}[ht]
\centering
\begin{tabular}{lccc}
\hline
& \#Palabras Sobre Region &\#Palabras en el resto de Argentina &Total \\ \hline
\# Palabras w &   k & K-k & K \\ 
\# Palabras $\neq$ w & n-k & N + k - n - K  & N - K \\ 
Total & n & N -n & N \\ \hline
\end{tabular}
\caption{Tabla de contingencia}
\label{tab:contingencia}
\end{table}


% Por lo tanto, sea  cantPalabrasW(Region) igual a la cantidad de ocurrencias observada de la palabra en la región a analizar.
% $$
% \begin{cases}
% H_0 :  x > cantPalabrasW(Region) \\
% H_1 : x \leq cantPalabrasW(Region)
% \end{cases}
% $$  
% siendo x la esperanza de la variable aleatoria que representa la cantidad de palabras exitosas en esa región.

% agregar gráficos de palabras comunes, con los parametros de la distribucion
% hacer un grafico de la distribución estimada (y la observada) de la cantidad de ocurrencias de la palabra 
En primer lugar hicimos el test estadístico sobre las palabras del conjunto de datos de desarrollo para ver resultados preliminares. El test lo realizamos sobre las 5000 palabras candidatas a ser contrastivas identificadas a través de nuestra métrica. Una vez realizado este test obtuvimos los p-valores de la figura \ref{fig:p-valores}. Debido a que realizamos múltiples test tuvimos que aplicarle una corrección para evitar falsos positivos. Decidimos utilizar la corrección de Bonferroni con $\alpha = 0.5$.
 %TODO: Hablar sobre test multiples y correcciones posibles, citar bonferroni


\begin{figure}[!ht]\centering
    \includegraphics[width=\linewidth]{images/pvaloresHipergeometrico.pdf}
    \caption{Gráfico de dispersión de los p-valores del test hipergeométrico sobre las primeras 5000 palabras del listado ordenado según la métrica del valor de la información.} 
    \label{fig:p-valores}   
\end{figure}


Ante los p-valores tan bajos, decidimos hacer el test estadístico sobre palabras que consideramos que no deberían tener una frecuencia muy variada en las distintas regiones. Realizamos el test para las palabras \{que, cuando, hola\} y estos también dieron p-valores menores a 0.001. Frente a esta situación investigamos las posibles causas de este fenómeno.

Un estudio de Adam Kilgariff titulado \textit{Language is never, ever, ever, random}\cite {kilgarriff2005language} analiza el uso de ${\chi}^2$ y log-likelihood test son problemáticos, ya que se basan en la suposición de que todas las muestras son estadísticamente independientes. 
El autor afirma que debido a que el lenguaje no es aleatorio (ya que hablamos y escribimos con una intencionalidad) y que la hipótesis nula de los test estadísticos asumen aleatoriedad, cuando los datos provienen de fenómenos lingüísticos sobre corpus la hipótesis nula nunca es cierta. 
Kilgariff agrega que cuando hay suficientes datos (casi) siempre podemos ser capaces de rechazar la hipótesis nula. Con esto, el autor no quiere decir que los tests estadísticos no sirven, incluso menciona que en estadística se realizan tests que suponen la aleatoriedad cuando el fenómeno a estudiar no presenta esa propiedad. El problema está en realizar estos tests cuando el fenómeno es tan arbitrario como el lenguaje y de esta manera se simplifica en demasía la hipótesis de aleatoriedad.
% llega a conclusiones que no aportan nada de información ya que la premisa del test estadístico es falsa. Por lo tanto, al ser falsa la premisa uno obtiene una implicación, en nuestro caso el resultado del test, trivialmente verdadera. 
% Hacer el grafico dependiendo de las frecuencias
% Comentar acerca del experimento en el que ve que para las palabras con mas frecuencia hay un mayor error. esto quizas explica porque las palabras con menos valor de la informacion tienen p-valores mas altos 
Kilgariff también hace un análisis sobre trabajos previos y comenta el caso en el que un estudio quería encontrar palabras con frecuencias significativas entre el inglés británico y el americano, representados por el Corpus Brown(inglés americano) y el Lancaster-Oslo-Bergen Corpus(inglés británico). Para cada palabra testearon la hipótesis nula la cual afirmaba que la diferencia entre las frecuencias sobre los dos corpus se debía a una fluctuación  aleatoria. El test lo hicieron a partir de muestras obtenidas aleatoriamente de los corpus mencionados. En este estudio se marcaron las palabras donde la hipótesis nula se rechazaba con distintos niveles de confianza. Las listas sugieren que la mayor parte de las palabras \textit{comunes} fueron marcadas, es decir que el test estadístico sugería que todas estas palabras tenían diferencias significativas en su uso. Esto que comenta Kilgariff es lo que tuvimos como resultado a partir de nuestro test hipergeométrico. Lo interesante es que el autor atribuye ese fenómeno a la esencia no aleatoria del lenguaje. Si bien sabíamos que al realizar el test hipergeométrico asumíamos que todas las palabras son estadísticamente independientes, pensamos que esta suposición no iba a afectar tanto los resultados como lo hizo.

Tanto Kilgariff \cite{kilgarriff2001comparing} como Paquot y Bestgen \cite{paquot2009distinctive} afirman que es posible representar los datos de manera diferente y hacer uso de otros tests asumiendo la independencia entre los textos y no entre las palabras. De esta manera se pueden observar la distribución de las palabras dentro de un corpus. En un estudio más reciente, Lijffijt et al. \cite{lijffijtsignificance} analiza algunas de estas alternativas para hacer un test de hipótesis en los cuales no se asume el modelo de \textit{bag-of-words}. Este modelo es el que calcula las frecuencias de las palabras sobre todos los textos, sin tener en cuenta la distribución de las frecuencias en cada uno.  En ese trabajo se analizaron el test t de Welch \cite{welch1947generalization}, el test de los rangos con signo de Wilcoxon(Wilcoxon rank sum), el de Bootrstrap y el de tiempo entre llegadas (inter-arrival time). Lijffijt explica que la diferencia entre estos tests con los que asumen el modelo de \textit{bag-of-words}( ${\chi}^2$ y log-likelihood test) reside en la representación de los datos, ergo la unidad de observación:

Para los tests que suponen los modelos de bolsa de palabras o \textit{bag-of-words}, los datos se representan en una tabla de contingencia de 2x2 y el número de muestras equivale al número de palabras en el corpus, mientras que en los otros cuatro tests, los datos son representados en una lista de frecuencias o una lista de \textit{tiempos de llegada}. En estos casos, el número de muestras es mucho menor que la cantidad de palabras en el corpus. Se destaca el número de muestras de los modelos ya que este número generalmente determina nuestro nivel de seguridad en relación a los valores estimados. Es por esto que los resultados experimentales muestran que los tests de modelos de \textit{bag-of-words} tienen una excesiva alta confianza en los valores estimados de la frecuencia media de las palabras, en el contexto de comparación estadística entre dos corpus. \cite{lijffijtsignificance}.
% TODO: CITAR

%TODO: agregar los resultados de las palabras del test hipergeometrico.
% las palabras raras, y el grafico de pvalores. a partir de eso comentar el trabajo de kilgariff 
% que habla de test de hipotesis, el error de asumir que las palabras vienen de un proceso aleatorio.

\subsection{Test t de Welch}
Basándonos en las propuestas de Lijffijt, decidimos utilizar el test de Welch. Este nos provee un valor de probabilidad para rechazar la hipótesis nula la cual afirma que las medias de las dos distribuciones son iguales. Sean $S$ y $T$ dos corpus y sea $q$ la palabra sobre la cual se va a hacer el test, sea $x_1$ la media de la frecuencia de la palabra $q$ sobre los textos de $S$, y sea $s_1$ la desviación estándar. Análogamente, sea $x_2$ la media de la frecuencia de $q$ en los textos $T$ y $s_2$ la desviación estándar. El estadístico $t$ se calcula con la ecuación \ref{eq:estadistico_welch}. Las suposiciones del test consisten en que todos los textos son estadísticamente independientes y que la media de las frecuencias proviene de una distribución normal. En nuestro caso, agrupamos todos los tuits de cada usuario representando un texto. De esta manera, cada provincia tiene alrededor de 900 textos formados por distintos usuarios \footnote{La cantidad de usuarios recolectados por cada provincia se encuentra detallada en la tabla \ref{tab:cantidades}}. Luego, el test es aplicado a cada palabra con las frecuencias entre dos corpus: uno está formado por todos los textos de los usuarios que provienen de las provincias en donde se cubre el 80\% de las ocurrencias, el otro consiste en los textos creados por usuarios del resto de las provincias. Notar que asumir la independencia entre todos los textos de un usuario de una provincia, es una suposición más débil que la de asumir independencia en todos los textos generados en una provincia como se hacía en el modelo de bolsa de palabras. 

\begin{equation}
\label{eq:estadistico_welch}
 t = \frac{x_1-x_2}{\sqrt{\frac{s_1^2}{\lvert S \rvert}+\frac{s_2^2}{\lvert T \rvert}}}  
\end{equation}

Realizando el test de Welch no queda claro si la distribución de los p-valores es la esperada. Ante esta posibilidad conjeturamos con que quizás la comparación entre en la hipótesis que formulamos en el análisis estadístico y la contrastividad dada por la métrica no es la más adecuada. Igualmente, no se replicó de manera idéntica el trabajo de Lifjjift et.al ya que nosotros cambiamos de corpus por cada palabra. La causa de esto es la que definimos una región compuesta de provincias en función de las frecuencias de la palabra en las distintas provincias argentinas. El análisis de significancia a partir de test estadísticos sobre fenómenos lingüísticos de corpus no es algo que este resuelto en la actualidad. Dejamos como trabajo futuro el cálculo de un  análisis estadístico comparable a nuestra métrica ya que está por fuera del alcance de esta tesis.

Sin embargo, a partir del test de Welch obtuvimos algunos resultados que dan un indicio de las virtudes de la métrica desarrollada, los cuales exponemos a continuación. Uno de ellos es la comparación de las tasas de rechazo de la hipótesis nula sobre las tres métricas desarrolladas, $I_u$ , $I_w$ y $I$ mencionadas en \ref{eq:iu}, \ref{eq:iw} y \ref{eq:ivalor} respectivamente.
Calculamos la taza de rechazo para las palabras en distintos intervalos del listado ordenado según las tres métricas elegidas: la métrica que tiene en cuenta la entropía de palabras, la valora la entropía de personas, y la que contiene a los dos factores. En la figura \ref{fig:rechazo_metricas} se muestran los resultados, donde la tasa de rechazo se calcula después de haber aplicado la corrección de Bonferroni para evitar falsos positivos. La métrica elegida, la cual tiene a ambos factores en consideración tiene una mejor taza de rechazo de la hipótesis nula en las palabras consideradas contrastivas y una menor taza de rechazo para el resto. 
Es importante notar que el test de Welch que realizamos tiene como muestras las distintas frecuencias de todos los usuarios en cada región. Por lo tanto es razonable obtener un resultado como este, en el cual las métricas que consideran la dispersión de las frecuencias sobre todos los usuarios para distinguir el nivel de contrastividad de la palabra.

\begin{figure}[!ht]\centering
  
    \includegraphics[width=\linewidth]{./images/rechazo_metricas.pdf}
    \caption{Resume la tasa de rechazo de la hipótesis nula en los distintos conjuntos de palabras, los cuales varían según el índice de estas en el listado ordenado según la métrica elegida.} 
    \label{fig:rechazo_metricas} 

\end{figure}

También es importante destacar que a medida que uno se aleja de las palabras más contrastivas de acuerdo a nuestra métrica, la tasa de rechaza es menor. Esto refleja el buen comportamiento de la métrica. Este resultado se puede observar también en la figura \ref{fig:pvalores_sinBonferroni} donde se detalla la distribución de los p-valores en el conjunto de las primeras 5000 palabras y el resto de los términos del listado.

\begin{figure}[!ht]\centering
  
    \includegraphics[width=\linewidth]{./images/pvalores_sinBonferroni.pdf}
    \caption{Distribución de p-valores (sin corrección de tests múltiples) en los distintos conjuntos de palabras. De izquierda a derecha se muestran los gráficos dependiendo de los índices de las palabras en el listado ordenado por la métrica del valor de la información: [0,5000], [20000,25000],[50000,55000]} 
    \label{fig:pvalores_sinBonferroni} 

\end{figure}



\chapter{Conclusiones y trabajo futuro}
\label{ch:conclusiones}

\subsection{Conclusiones}
En el presente trabajo recolectamos un conjunto de datos de texto de la Argentina a través de la API de Twitter. Este conjunto lo dividimos en dos, un conjunto para desarrollar una métrica que indique el valor contrastivo de una palabra. El segundo conjunto de datos, independiente del primero en cuanto a usuarios y al período temporal de los textos lo utilizamos para hacer un test estadístico para corroborar que las palabras detectadas como contrastivas según nuestra métrica, no estaban sobrerepresentadas en el conjunto de desarrollo.

La métrica que realizamos usaba la entropía para medir la variación de la cantidad de ocurrencias y de la cantidad de usuarios que la utilizaban en las distintas provincias del país. Creamos un listado de palabras ordenado según el valor de la información calculada, y a partir de ella filtramos las palabras de forma manual eliminando los términos que no tengan valor lingüístico, como los nombres propios (como los nombres de personas, o de lugares). Varias palabras del listado, como \textit{mitaí, anga,angau} resultaron muy interesantes a nivel lingüístico


Sobre estas palabras realizamos el test estadístico. 

% Comentar que la metrica evidentemente detecta palabras contrastivas e interesantes a nivel linguistico.
% Hablar tambien de los problemas encontrados.
\subsection{Trabajo Futuro}

%Separar los nuevos desafios de las mejoras tecnicas a realizar sobre este trabajo
Uno de los desafíos que quedan para hacer es el de clasificar las regiones en clusters, obteniendo así las regiones dialectales. De esta manera se podría ver la vigencia de las regiones descriptas por Vidal de Battini. % Hacer referencia a la sección de la introdcucción que se habla de eso

El proceso de normalización se podría mejorar para tener una mejor precisión de las palabras utilizadas. También se podría agregar un sistema de reconocimiento de nombres de entidades para filtrar los nombres propios de manera tal que el listado de palabras contrastivas tengan menos términos sin interés lingüístico.

Por otro lado, este trabajo se podría realizar sobre todo el conjunto de países hispanoparlantes, de modo tal que se puedan hacer comparaciones entre los mismos y comparar las variaciones entre regiones más grandes.\\

También queda por hacer un análisis sintáctico de las oraciones, y un análisis estadístico de bigramas.
Otro desafío es el de analizar los tuits modelados por cadenas de markov (analizando así bigramas y n-gramas), pudiendo generar un bot que cree tuits, este bot podría ser parametrizado de modo tal que genere textos, teniendo en cuenta únicamente los tuits de determinada región.



\chapter{Apéndice}
\label{ch:apendice}
\subsection{La entropía como médida del desorden}
\label{sub:entropiaShannon}

Para ver la cantidad de información que nos aporta cada palabra se hará una introducción a la teoría de la información, especificamente
los conceptos que introdujo Claude Shannon\cite{shannon2001mathematical}.
Para entender estos conceptos es útil tener una descripción matemática del mecanismo que genera la información. Para eso se define a 
la \textit{fuente} que emite señales de un alfabeto $ S = \{s_1, s_2, \dots\, s_q\}$ de acuerdo a una función de probabilidad fija.
Si la fuente emite señáles estadísticamente independientes decimos que es una \textit{fuente de memoria nula} y un símbolo $s$ está completamente determinado por el alfabeto $S$ y las probabilidades:
$P(s_1)\,P(s_2)\, \dots\, P(s_q)$

Sea X una variable aleatoria discreta con posibles valores $\{x_1, x_2, \dots\, x_q\}$ y una función de probabilidad P(X), luego:
${\displaystyle \mathrm {H} (X)=\mathrm {E} [\mathrm {I} (X)]=\mathrm {E} [-\log(\mathrm {P} (X))].}$
donde X es una variable aleatoria con posibles valores $\{x_1, ... , x_n\}$ y P es una función de probabilidad.

Los símbolos con menor probabilidad son los que aportan más información. Esto va de la mano con nuestra intuición ya que si entendemos a los símbolos como palabras de un texto, las palabras más utilizadas como \textit{de} o \textit{que} aportan menos información que la palabra \textit{celular}. 
Observaciones:
\begin{itemize}
    \item La entropía es máxima cuando los eventos de X son equiprobables. En este caso, si hay n eventos con una probabilidad de $\frac{1}{n}$ cada uno , el valor de la entropía es de $\log n$.
    \item La entropía es 0 si y solo sí todas las probabilidades son 0 a excepeción de una con probabilidad igual a la unidad. 
\end{itemize}

Dado que la entropía es máxima cuando los eventos de X son equiprobables, se suele decir que es una médida del desorden.



%%%% BIBLIOGRAFIA
\chapter{Bibliografía}
\backmatter

\bibliographystyle{alpha}
\bibliography{tesis}
\end{document}